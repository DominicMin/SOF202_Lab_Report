\documentclass{xum_review}
%------------------------------------------------------------
% 文档类xum_review已包含所需的大部分包和代码环境定义
% 这里只需要添加文档特定的设置
%------------------------------------------------------------

\begin{document}

\includepdf[pages=-]{cp.pdf}

\pagestyle{plain} % 设置目录页面无页码,但保留页眉
\tableofcontents
\clearpage % 使用clearpage确保目录完整显示
\pagestyle{fancy} % 恢复正常页码样式
\fancyfoot[C]{Page \thepage\ of \pageref{LastMainPage}} % 使用自定义的最后一页标签
\setcounter{page}{1}

\addcontentsline{toc}{section}{Project Gantt Chart and Milestones}
\includepdf[pages=-, fitpaper=true]{figure/gantt.pdf}


\section{Requirement Analysis}

\subsection{Problem Statement}

The university sports complex currently relies on outdated methods such as
manual registration and spreadsheets to manage equipment, facilities, coaches,
and bookings. This has led to issues including double bookings, missing records,
and difficulties in tracking equipment maintenance.

\subsection{Objectives}

Develop a conceptual database system to:

\begin{itemize}
    \item Maintain a clear catalog of facilities and equipment.
    \item Allow members (students, faculty, and external visitors) to register and make bookings.
    \item Enable coaches to be assigned to training sessions.
    \item Prevent booking conflicts.
    \item Track maintenance schedules for equipment and facilities.
\end{itemize}

\subsection{Functional Requirements}

\begin{enumerate}
    \item The system shall manage member registration with different processes
    based on user type:
    \begin{itemize}
        \item \textbf{Students and Staff:} Automatically become \texttt{Member}
        upon enrollment or employment at the university. Their
        \texttt{Student\_ID} or \texttt{Staff\_ID} serves as proof of
        affiliation, and corresponding \texttt{Member}, \texttt{Student}, or
        \texttt{Staff} records are created directly in the system.
        \item \textbf{External Visitors:} Must submit a
        \texttt{Visitor\_Application} for approval. Upon approval by
        administrative staff, \texttt{Member} and \texttt{External\_Visitor}
        records are created.
    \end{itemize}
    
    \item The system shall enable members to book available facilities during
    specific time slots.
    
    \item Before confirming any new booking, the system must
    automatically check for conflicts with the selected facility and time slot.
    If a conflict exists (double booking), the system must reject the request
    and notify the user.
    
    \item The system shall provide administrators with a real-time interface to
    track facility and equipment usage.
    
    \item The system shall allow members to view their current, future, and
    historical booking records after logging in.
    
    \item The system shall track and record equipment maintenance schedules and
    maintenance history.
    
    \item The system shall record the timestamp, status, and member information
    for each booking transaction.
    
    \item The system shall assign training sessions based on coach availability
    and specialization.
    
    \item The system shall allow administrators to add, update, or delete
    facility and equipment records.
\end{enumerate}

\subsection{Non-Functional Requirements}

\begin{enumerate}
    \item \textbf{Performance:} The system shall respond to all booking, update,
    or query requests within 3 seconds under normal load, and shall be capable
    of handling at least 200 members browsing and booking facilities
    simultaneously.
    
    \item \textbf{Security:} The system must encrypt all member information
    using industry-standard algorithms (such as \texttt{SHA-256}), and implement
    Role-Based Access Control (\texttt{RBAC}) to restrict unauthorized data
    access.
    
    \item \textbf{Usability:} The system shall provide a user-friendly and
    intuitive interface accessible via both desktop and mobile browsers,
    featuring clear navigation menus, input validation prompts, and booking
    confirmation dialogs to minimize user errors. Additionally, the system
    shall support multi-language interfaces (such as English and Chinese)
    for display.
    
    \item \textbf{Reliability:} The system shall perform daily automated backups
    to prevent data loss.
    
    \item \textbf{Auditability:} The system must log all administrator
    operations (create, read, update, delete) on facilities, equipment, and
    member records, generating tamper-proof audit logs for future security
    audits and accountability tracking.
\end{enumerate}

\subsection{Assumptions}\label{assumptions}

\begin{enumerate}
    \item We assume that member registration follows different processes based
    on affiliation:
    \begin{itemize}
        \item \textbf{Students and Faculty (Staff):} Automatically become
        members upon university enrollment or employment. Their university
        \texttt{Student\_ID} or \texttt{Staff\_ID} is used to create
        corresponding records in the \texttt{Member}, \texttt{Student}, or
        \texttt{Staff} tables. They authenticate using their existing
        university credentials (ID and password).
        \item \textbf{External Visitors:} Must submit a
        \texttt{Visitor\_Application} and wait for administrative approval
        before becoming members. Upon approval, \texttt{Member} and
        \texttt{External\_Visitor} records are created, and they can register
        a new account with login credentials.
    \end{itemize}
    
    \item We assume that the system must enforce specific booking rules
    established by the sports complex administration:
    \begin{itemize}
        \item Each member may have a maximum of two active pending bookings at
        any given time.
        \item Each member's single booking session cannot exceed three hours in
        total duration.
        \item Members may book facilities up to one week in advance.
    \end{itemize}
    
    \item We assume that the system will be developed as a responsive web
    application, accessible to users directly through browsers on desktop
    computers and mobile devices.
    
    \item We assume that an \texttt{administrator} role will exist with the
    highest level of privileges, including: editing facility and equipment
    catalogs, managing all member accounts, assigning coaches, and scheduling
    maintenance.
\end{enumerate}

\section{Conceptual Design}

\subsection{Entity and Attribute Identification}

Based on the requirements analysis, the following entities have been identified
for the University Sports Complex Management System:

\subsubsection{Member (Superclass Entity)}

\textbf{Description:} Represents individuals who can use the sports complex
facilities. Members are created through different processes:
\begin{itemize}
    \item \textbf{Students and Staff:} Automatically registered as members upon
    university enrollment or employment, using their \texttt{Student\_ID} or
    \texttt{Staff\_ID}.
    \item \textbf{External Visitors:} Become members only after submitting and
    receiving approval for a \texttt{Visitor\_Application}.
\end{itemize}

\textbf{Attributes:}
\begin{itemize}
    \item \texttt{Member\_ID} (\texttt{INT}, \texttt{PK}) - Unique identifier for each member
    \item \texttt{First\_Name} (\texttt{VARCHAR(50)}) - Member's first name
    \item \texttt{Last\_Name} (\texttt{VARCHAR(50)}) - Member's last name
    \item \texttt{Registration\_Date} (\texttt{DATE}) - Date when the member registered
    \item \texttt{Membership\_Status} (\texttt{ENUM}: Active, Inactive,
    Pending\_Approval) - Current membership status
    \item \texttt{Contact\_Details} (Composite attribute) - Contact information
    \begin{itemize}
        \item \texttt{Phone\_Number} (\texttt{VARCHAR(20)}, Multi-valued) - Contact phone
        number(s)
        \item \texttt{Email\_Address} (\texttt{VARCHAR(100)}, Multi-valued) - Contact email
        address(es)
    \end{itemize}
\end{itemize}

\textbf{Specializations:}
\begin{itemize}
    \item \textbf{Student:} \texttt{Student\_ID} (\texttt{VARCHAR(20)}) - University
    student identification number
    \item \textbf{Staff:} \texttt{Staff\_ID} (\texttt{VARCHAR(20)}) - University staff
    identification number
    \item \textbf{External\_Visitor:} \texttt{IC\_Number} (\texttt{VARCHAR(20)}) -
    National identification card number
\end{itemize}

\subsubsection{Facility}

\textbf{Description:} Represents physical sports facilities available for
booking.

\textbf{Attributes:}
\begin{itemize}
    \item \texttt{Facility\_ID} (\texttt{INT}, \texttt{PK}) - Unique identifier for each facility
    \item \texttt{Facility\_Name} (\texttt{VARCHAR(100)}) - Name of the facility
    \item \texttt{Type} (\texttt{VARCHAR(50)}) - Type of facility (e.g., Basketball
    Court, Swimming Pool)
    \item \texttt{Status} (\texttt{ENUM}: Available, Under\_Maintenance, Unavailable) -
    Current availability status
    \item \texttt{Capacity} (\texttt{INT}) - Maximum number of users allowed
    \item \texttt{Building} (\texttt{VARCHAR(50)}) - Building name (part of composite
    Location attribute)
    \item \texttt{Floor} (\texttt{INT}) - Floor number (part of composite Location
    attribute)
    \item \texttt{Room\_Number} (\texttt{VARCHAR(20)}) - Room identifier (part of
    composite Location attribute)
\end{itemize}

\subsubsection{Equipment}

\textbf{Description:} Represents sports equipment that can be reserved by
members.

\textbf{Attributes:}
\begin{itemize}
    \item \texttt{Equipment\_ID} (\texttt{INT}, \texttt{PK}) - Unique identifier for each
    equipment type
    \item \texttt{Equipment\_Name} (\texttt{VARCHAR(100)}) - Name of the equipment
    \item \texttt{Type} (\texttt{VARCHAR(50)}) - Category of equipment
    \item \texttt{Status} (\texttt{ENUM}: Available, In\_Use, Under\_Maintenance, Damaged) -
    Current status
    \item \texttt{Total\_Quantity} (\texttt{INT}) - Total number of items available
    \item \texttt{Available\_Quantity} (\texttt{INT}, Derived) - Not stored in the database.
    Calculated as: Total\_Quantity - (quantity in active reservations + count of active
    maintenance records)
\end{itemize}

\subsubsection{Reservation (Superclass Entity)}

\textbf{Description:} Represents a time slot reservation for a facility.

\textbf{Attributes:}
\begin{itemize}
    \item \texttt{Reservation\_ID} (\texttt{INT}, \texttt{PK}) - Unique identifier for each
    reservation
    \item \texttt{Facility\_ID} (\texttt{INT}, \texttt{FK}) - References Facility
    \item \texttt{Reservation\_Date} (\texttt{DATE}) - Date of the reservation
    \item \texttt{Start\_Time} (\texttt{TIME}) - Starting time of the reservation
    \item \texttt{End\_Time} (\texttt{TIME}) - Ending time of the reservation
\end{itemize}

\textbf{Specializations:}
\begin{itemize}
    \item \textbf{Booking:} Member reservations with \texttt{Member\_ID} (\texttt{INT},
    \texttt{FK}) and \texttt{Booking\_Status} (\texttt{ENUM}: Pending, Confirmed, Cancelled)
    \item \textbf{Training\_Session:} Coach-led sessions with \texttt{Coach\_ID}
    (\texttt{INT}, \texttt{FK}) and \texttt{Max\_Capacity} (\texttt{INT})
\end{itemize}

\subsubsection{Coach}

\textbf{Description:} Represents sports coaches who conduct training sessions.

\textbf{Attributes:}
\begin{itemize}
    \item \texttt{Coach\_ID} (\texttt{INT}, \texttt{PK}) - Unique identifier for each coach
    \item \texttt{First\_Name} (\texttt{VARCHAR(50)}) - Coach's first name
    \item \texttt{Last\_Name} (\texttt{VARCHAR(50)}) - Coach's last name
    \item \texttt{Sport\_Type} (\texttt{VARCHAR(50)}) - Sport specialization (part of
    composite Specialization attribute)
    \item \texttt{Level} (\texttt{VARCHAR(20)}) - Certification level (part of composite
    Specialization attribute)
    \item \texttt{Contact\_Details} (Composite attribute) - Contact information
    \begin{itemize}
        \item \texttt{Phone\_Number} (\texttt{VARCHAR(20)}, Multi-valued) - Contact phone
        number(s)
        \item \texttt{Email\_Address} (\texttt{VARCHAR(100)}, Multi-valued) - Contact email
        address(es)
    \end{itemize}
\end{itemize}

\subsubsection{Maintenance}

\textbf{Description:} Represents maintenance records for facilities and
equipment.

\textbf{Attributes:}
\begin{itemize}
    \item \texttt{Maintenance\_ID} (\texttt{INT}, \texttt{PK}) - Unique identifier for each
    maintenance record
    \item \texttt{Scheduled\_Date} (\texttt{DATE}) - Planned maintenance date
    \item \texttt{Completion\_Date} (\texttt{DATE}) - Actual completion date
    \item \texttt{Status} (\texttt{ENUM}: Scheduled, In\_Progress, Completed, Cancelled)
    - Maintenance status
    \item \texttt{Description} (\texttt{TEXT}) - Details of maintenance work
    \item \texttt{Facility\_ID} (\texttt{INT}, \texttt{FK}, Optional) - References Facility (XOR
    with Equipment\_ID)
    \item \texttt{Equipment\_ID} (\texttt{INT}, \texttt{FK}, Optional) - References Equipment (XOR
    with Facility\_ID). Each maintenance record represents maintenance of one equipment
    instance; multiple equipment items require multiple maintenance records
\end{itemize}

\subsubsection{Visitor\_Application}

\textbf{Description:} Represents membership registration applications from
prospective external visitors who are not yet members of the system. Upon
approval, a \texttt{Member} and \texttt{External\_Visitor} record will be
created.

\textbf{Attributes:}
\begin{itemize}
    \item \texttt{Application\_ID} (\texttt{INT}, \texttt{PK}) - Unique identifier for each
    application
    \item \texttt{Name} (Composite attribute) - Applicant's full name
    \begin{itemize}
        \item \texttt{First\_Name} (\texttt{VARCHAR(50)}) - First name
        \item \texttt{Last\_Name} (\texttt{VARCHAR(50)}) - Last name
    \end{itemize}
    \item \texttt{Contact\_Details} (Composite attribute) - Applicant's contact information
    \begin{itemize}
        \item \texttt{Phone\_Number} (\texttt{VARCHAR(20)}, Multi-valued) - Contact phone
        number(s)
        \item \texttt{Email\_Address} (\texttt{VARCHAR(100)}, Multi-valued) - Contact email
        address(es)
    \end{itemize}
    \item \texttt{IC\_Number} (\texttt{VARCHAR(20)}) - Applicant's identification card number
    \item \texttt{Application\_Date} (\texttt{DATE}) - Date application was submitted
    \item \texttt{Status} (\texttt{ENUM}: Pending, Approved, Rejected) -
    Application status
    \item \texttt{Approved\_By} (\texttt{INT}, \texttt{FK}) - References Staff (administrator who
    reviews the application)
    \item \texttt{Approval\_Date} (\texttt{DATE}) - Date of approval/rejection
    \item \texttt{Reject\_Reason} (\texttt{TEXT}) - Reason for rejection (if applicable)
\end{itemize}

\textbf{Note:} This entity represents applications from \emph{prospective}
external visitors. The applicant is not yet a \texttt{Member} when submitting
the application. After approval, the staff creates corresponding
\texttt{Member} and \texttt{External\_Visitor} records based on the
application information.

\subsubsection{Weak Entities and Associative Entities}

Additional entities identified to support relationships:

\begin{itemize}
    \item \textbf{Reservation\_Equipments} (Weak Entity): Links reservations
    with equipment
    \begin{itemize}
        \item Composite \texttt{PK}: (\texttt{Reservation\_ID}, \texttt{Equipment\_ID})
        \item \texttt{Quantity} (\texttt{INT}) - Number of items reserved
    \end{itemize}
    
    \item \textbf{Session\_Enrollment} (Weak Entity): Links members with
    training sessions
    \begin{itemize}
        \item Composite \texttt{PK}: (\texttt{Reservation\_ID}, \texttt{Member\_ID})
    \end{itemize}
    
    \item \textbf{Member\_Phone} (Weak Entity): Stores multi-valued phone
    numbers
    \begin{itemize}
        \item Composite \texttt{PK}: (\texttt{Member\_ID}, \texttt{Phone\_Number})
    \end{itemize}
    
    \item \textbf{Member\_Email} (Weak Entity): Stores multi-valued email
    addresses
    \begin{itemize}
        \item Composite \texttt{PK}: (\texttt{Member\_ID}, \texttt{Email\_Address})
    \end{itemize}
    
    \item \textbf{Coach\_Phone} (Weak Entity): Stores coach phone numbers
    \begin{itemize}
        \item Composite \texttt{PK}: (\texttt{Coach\_ID}, \texttt{Phone\_Number})
    \end{itemize}
    
    \item \textbf{Coach\_Email} (Weak Entity): Stores coach email addresses
    \begin{itemize}
        \item Composite \texttt{PK}: (\texttt{Coach\_ID}, \texttt{Email\_Address})
    \end{itemize}
\end{itemize}

\subsection{Enhanced Entity-Relationship Diagram (EERD)}

The complete EERD illustrating all entities, attributes, relationships, and
constraints is shown below:

\includepdf[pages=-, fitpaper=true]{figure/eerd.pdf}

\section{Logical Design}\label{logical-design-task-3-from-eerd-to-relational-schema}

Based on the Enhanced Entity-Relationship Diagram (EERD), this section
provides a systematic relational schema design for Scenario 1
"University Sports Facility Management System" and explains the
conversion principles from conceptual model to relational model. The
overall design follows these fundamental guidelines:

\begin{enumerate}
\def\labelenumi{\arabic{enumi}.}
\item
  \textbf{Strong Entity → Basic Relational Schema}\\ Each strong entity
  (such as \texttt{Member}, \texttt{Facility}, \texttt{Equipment},
  \texttt{Coach}, etc.) corresponds to a relational schema.
\item
  \textbf{Weak Entity / Associative Entity → Relational Schema with
  Composite Primary Key}\\ Such as \texttt{Session\_Enrollment},
  \texttt{Reservation\_Equipments}, etc., where the primary key is
  typically composed of the primary keys of related entities.
\item
  \textbf{1:N Relationship → Foreign Key at N-side}\\ For example, the
  relationship between \texttt{Member} and \texttt{Booking} is
  implemented by introducing \texttt{Member\_ID} as a foreign key in the
  \texttt{Booking} table.
\item
  \textbf{N:M Relationship → Independent Association Table}\\ For
  instance, member enrollment in training sessions and equipment usage
  in reservations are implemented through tables with composite primary
  keys.
\item
  \textbf{Generalization/Specialization}\\ For \texttt{Member} with
  \texttt{Student}/\texttt{Staff}/\texttt{External\_Visitor}, a
  "superclass + subclass" table structure is adopted; for
  \texttt{Reservation} with \texttt{Booking}/\texttt{Training\_Session},
  a primary key pushdown subclass structure is used.
\item
  \textbf{Handling Composite Attributes and Multi-valued Attributes}
  \begin{itemize}
  \item
    Composite attributes (such as \texttt{Name}, \texttt{Location},
    \texttt{Contact\_Details}) are decomposed into multiple atomic
    attributes;
  \item
    Multi-valued attributes (such as multiple phone numbers and emails)
    are converted into independent association tables.
  \end{itemize}
\item
  \textbf{Derived Attributes Not Stored}\\ For example, equipment's
  \texttt{Available\_Quantity} is calculated from total quantity and
  allocated quantities in reservations, and is not stored in base tables
  to avoid redundancy and inconsistency.
\end{enumerate}

Under the guidance of these principles, the following relational schema
design is obtained.

\subsection{Personnel and Membership Management Related
Schemas}\label{personnel-and-membership-management-related-schemas}

\subsubsection{Member Superclass and Subclasses}\label{member-superclass-and-subclasses-student-staff-external_visitor}

\begin{itemize}
\item
  \textbf{Member (Superclass)}

  \texttt{Member(Member\_ID\ PK,\\
  \hspace*{2em} First\_Name,\\
  \hspace*{2em} Last\_Name,\\
  \hspace*{2em} Registration\_Date,\\
  \hspace*{2em} Membership\_Status)}

  \begin{itemize}
  
  \item
    \texttt{First\_Name,\ Last\_Name} are decomposed from the composite
    attribute \texttt{Name};
  \item
    \texttt{Membership\_Status} is used to distinguish states such as
    Active / Inactive / Pending\_Approval.
  \end{itemize}
\item
  \textbf{Student (Subclass)}

  \texttt{Student(Member\_ID\ PK,\ Student\_ID,\ FK\ →\ Member(Member\_ID))}
\item
  \textbf{Staff (Subclass)}

  \texttt{Staff(Member\_ID\ PK,\ Staff\_ID,\ FK\ →\ Member(Member\_ID))}
\item
  \textbf{External\_Visitor (Subclass)}

  \texttt{External\_Visitor(Member\_ID\ PK,\ IC\_Number,\ FK\ →\ Member(Member\_ID))}
\end{itemize}

\begin{quote}
The "superclass primary key pushdown" strategy is adopted: each
subclass's \texttt{Member\_ID} is both the primary key of the table and
a foreign key referencing \texttt{Member}, consistent with the
specialization structure in EERD, while ensuring that any member belongs
to at most one subclass (mutual exclusion).
\end{quote}

\subsubsection{Member\_Phone and Member\_Email: Multi-valued Contact
Information}\label{member_phone-and-member_email-multi-valued-contact-information}

In the conceptual design, \texttt{Member} has a composite attribute 
\texttt{Contact\_Details} containing two multi-valued sub-attributes: 
\texttt{Phone\_Number} and \texttt{Email\_Address}. In relational schema, 
each multi-valued sub-attribute is converted to a separate weak entity table.

\begin{itemize}
\item
  \textbf{Member\_Phone}
  
  \texttt{Member\_Phone(Member\_ID\ PK,\ Phone\_Number\ PK,\ FK\ →\ Member(Member\_ID))}

  Corresponds to the multi-valued sub-attribute \texttt{Phone\_Number} of the 
  composite attribute \texttt{Contact\_Details}. It allows one member to 
  register multiple phone numbers. The composite primary key 
  \texttt{(Member\_ID,\ Phone\_Number)} prevents duplicate insertion of 
  the same phone number for a member.

\item
  \textbf{Member\_Email}
  
  \texttt{Member\_Email(Member\_ID\ PK,\ Email\_Address\ PK,\ FK\ →\ Member(Member\_ID))}

  Corresponds to the multi-valued sub-attribute \texttt{Email\_Address} of the 
  composite attribute \texttt{Contact\_Details}. It allows one member to 
  register multiple email addresses. The composite primary key 
  \texttt{(Member\_ID,\ Email\_Address)} prevents duplicate insertion of 
  the same email address for a member.
\end{itemize}

\subsubsection{Visitor\_Application}\label{visitor-application}
  
  \texttt{Visitor\_Application(Application\_ID\ PK,\\
  \hspace*{2em}First\_Name,\\
  \hspace*{2em}Last\_Name,\\
  \hspace*{2em}IC\_Number,\\
  \hspace*{2em}Application\_Date,\\
  \hspace*{2em}Status,\\
  \hspace*{2em}Approved\_By\ FK\ →\ Staff(Member\_ID),\\
  \hspace*{2em}Approval\_Date,\\
  \hspace*{2em}Reject\_Reason,\\
  \hspace*{2em}Created\_Member\_ID\ FK\ →\ Member(Member\_ID))}

  \begin{itemize}
  
  \item
    In the conceptual design, \texttt{Visitor\_Application} has a composite
    attribute \texttt{Name} (containing \texttt{First\_Name} and
    \texttt{Last\_Name}) and a composite attribute \texttt{Contact\_Details}
    (containing multi-valued sub-attributes \texttt{Phone\_Number} and
    \texttt{Email\_Address}). In the relational schema, composite attributes
    are flattened: \texttt{First\_Name} and \texttt{Last\_Name} become regular
    columns, while the multi-valued sub-attributes are converted to separate
    weak entity tables: \texttt{Visitor\_Application\_Phone} and
    \texttt{Visitor\_Application\_Email}.
  \item
    \texttt{IC\_Number} stores the applicant's identification card number,
    which can be used to verify identity and prevent duplicate applications.
  \item
    \texttt{Status} records the application status (Pending, Approved, or Rejected).
  \item
    \texttt{Approved\_By} is a foreign key referencing
    \texttt{Staff(Member\_ID)}, indicating the staff member who reviewed
    the application.
  \item
    \texttt{Created\_Member\_ID} is an optional foreign key that links to the
    \texttt{Member} record created after approval. This field is \texttt{NULL}
    for pending or rejected applications, and is set when the application is
    approved and a new member is created. This allows tracking which application
    led to which member registration.
  \end{itemize}
  
\subsubsection{Visitor\_Application\_Phone and Visitor\_Application\_Email}

\texttt{Visitor\_Application\_Phone(Application\_ID\ PK,\ Phone\_Number\ PK,\\
\hspace*{2em}FK\ →\ Visitor\_Application(Application\_ID))}

\texttt{Visitor\_Application\_Email(Application\_ID\ PK,\ Email\_Address\ PK,\\
\hspace*{2em}FK\ →\ Visitor\_Application(Application\_ID))}

These two weak entities convert the multi-valued sub-attributes of the
\texttt{Contact\_Details} composite attribute. They allow each application
to have multiple contact phone numbers and email addresses. The composite
primary keys prevent duplicate phone numbers or email addresses for the
same application.

\subsection{Facility, Equipment, and Reservation Related
Schemas}\label{facility-equipment-and-reservation-related-schemas}

\subsubsection{Sports Facility}\label{facility-sports-facility}


  \texttt{Facility(Facility\_ID\ PK,\\
  \hspace*{2em} Facility\_Name,\\
  \hspace*{2em} Type,\ Status,\\
  \hspace*{2em} Capacity,\\
  \hspace*{2em} Building,\\
  \hspace*{2em} Floor,\\
  \hspace*{2em} Room\_Number)}

  \begin{itemize}
  
  \item
    \texttt{Building\ /\ Floor\ /\ Room\_Number} are decomposed from the
    composite attribute \texttt{Location}, facilitating queries and
    displays based on building, floor, and room;
  \item
    \texttt{Status} reflects the current availability status of the
    facility, such as AVAILABLE / UNDER\_MAINTENANCE / CLOSED;
  \item
    \texttt{Capacity} describes the maximum occupancy of the facility.
  \end{itemize}

\subsubsection{Equipment}\label{equipment-equipment}


  \texttt{Equipment(Equipment\_ID\ PK,\ Equipment\_Name,\ Type,\ Status,\ Total\_Quantity)}

  \begin{itemize}
  
  \item
    \texttt{Total\_Quantity} is the total number of equipment items;
  \item
    \texttt{Available\_Quantity} is a derived attribute, not stored in
    the database. It is calculated dynamically as:
    \texttt{Available\_Quantity = Total\_Quantity - (quantity in active reservations
    + count of active maintenance records)}. Since each maintenance record
    represents one equipment instance, we count maintenance records rather
    than sum quantities. This ensures real-time accuracy without data redundancy.
  \item
    \texttt{Status} indicates equipment availability status (Available,
    In\_Use, Under\_Maintenance, Damaged), and combined with the
    \texttt{Maintenance} table provides complete maintenance history.
  \end{itemize}

\subsubsection{Reservation: Unified Time
Granularity}\label{reservation-unified-time-granularity}

  \texttt{Reservation(Reservation\_ID\ PK,\\
  \hspace*{2em} Facility\_ID\ FK\ →\ Facility(Facility\_ID),\\
  \hspace*{2em} Reservation\_Date,\\
  \hspace*{2em} Start\_Time,\\
  \hspace*{2em} End\_Time)}

  \begin{itemize}
  \item
    The Reservation entity is used to uniformly represent the fact that
    "a facility is occupied during a specific time period on a specific
    date", providing a unified time granularity for member bookings and
    training sessions. Note that \texttt{Maintenance} is modeled as an
    independent entity, not as a subclass of \texttt{Reservation}, but
    application-level logic (e.g., SQL queries, triggers) can check for
    active maintenance records to prevent conflicting reservations.
  \item
    In EERD, \texttt{Reservation} (superclass) has a N:1 relationship with
    \texttt{Facility}. Therefore, \texttt{Facility\_ID} is defined at the
    superclass level, allowing all subclasses (\texttt{Booking} and
    \texttt{Training\_Session}) to inherit this association. Both member
    bookings and coach-led training sessions require facility allocation,
    so this design ensures consistency and avoids redundancy.
  \end{itemize}


\subsubsection{Booking: Member
Reservation}\label{booking-member-reservation}

  \texttt{Booking(Reservation\_ID\ PK,\ Member\_ID\ FK\ →\ Member(Member\_ID),\ Booking\_Status)}

  \begin{itemize}
  
  \item
    \texttt{Reservation\_ID} serves as both Booking's primary
    key and foreign key referencing \texttt{Reservation}, reflecting
    that \texttt{Booking} is a subclass of \texttt{Reservation};
  \item
    In the EERD, members create reservations for facilities. In the relational
    schema, this is implemented by storing \texttt{Member\_ID} in the
    \texttt{Booking} subtype, which is in a 1:1 relationship with
    \texttt{Reservation} for member-initiated reservations. This design still
    allows one member to have multiple bookings while keeping the reservation
    supertype reusable for other purposes (such as \texttt{Training\_Session}).
  \item
    \texttt{Booking\_Status} records the booking status (Pending /
    Confirmed / Cancelled), facilitating front-end and back-end
    management.
  \end{itemize}

This structure ensures:

\begin{itemize}

\item
  Each Reservation corresponds to at most one Booking record (otherwise
  primary key conflict occurs), ensuring the same time period is not
  "claimed" by multiple members;
\item
  If a Reservation has no corresponding Booking, it means the time
  period has not been booked by members and can be used for other
  purposes (such as training sessions).
\end{itemize}

\subsubsection{Reservation\_Equipments (Week Entity)}\label{reservation_equipments-nm-relationship-between-reservation-and-equipment}

  \texttt{Reservation\_Equipments(\\
    \hspace*{2em} Reservation\_ID\ PK, \\ 
    \hspace*{2em} Equipment\_ID\ PK, \\
    \hspace*{2em} Quantity, \\ 
    \hspace*{2em} FK\ →\ Reservation(Reservation\_ID), \\ 
    \hspace*{2em} FK\ →\ Equipment(Equipment\_ID))}

  \begin{itemize}
  \def\labelenumi{\arabic{enumi}.}
  
  \item
    This table is an associative weak entity between Reservation and
    Equipment, describing the demand for different equipment in a
    reservation;
  \item
    The composite primary key \texttt{(Reservation\_ID,\ Equipment\_ID)}
    prevents duplicate registration of the same equipment under the same
    reservation;
  \item
    \texttt{Quantity} represents the demand quantity for the
    corresponding equipment in this reservation, which is the core field
    for subsequent calculation of available quantities.
  \end{itemize}

\subsection{Coach and Training Session Related
Schemas}\label{coach-and-training-session-related-schemas}

\subsubsection{Coach, Coach\_Phone, and Coach\_Email}\label{coach-coach_phone-and-coach_email}

\begin{itemize}
\item
  \textbf{Coach}
  
  \texttt{Coach(Coach\_ID\ PK,\ First\_Name,\ Last\_Name,\ Sport\_Type,\ Level)}

  \texttt{Sport\_Type,\ Level} are decomposed from the composite
  attribute \texttt{Specialization}, supporting queries and scheduling
  by sport category and coach level.
  
\item
  \textbf{Coach\_Phone and Coach\_Email}
  
  Similar to \texttt{Member}, \texttt{Coach} has a composite attribute 
  \texttt{Contact\_Details} containing two multi-valued sub-attributes: 
  \texttt{Phone\_Number} and \texttt{Email\_Address}. Each multi-valued 
  sub-attribute is converted to a separate weak entity table.
  
  \texttt{Coach\_Phone(Coach\_ID\ PK,\ Phone\_Number\ PK,\ FK\ →\ Coach(Coach\_ID))}

  Corresponds to the multi-valued sub-attribute \texttt{Phone\_Number} of the 
  composite attribute \texttt{Contact\_Details}. It allows one coach to 
  register multiple phone numbers. The composite primary key 
  \texttt{(Coach\_ID,\ Phone\_Number)} prevents duplicate phone numbers 
  for the same coach.

\item
  \texttt{Coach\_Email(Coach\_ID\ PK,\ Email\_Address\ PK,\ FK\ →\ Coach(Coach\_ID))}

  Corresponds to the multi-valued sub-attribute \texttt{Email\_Address} of the 
  composite attribute \texttt{Contact\_Details}. It allows one coach to 
  register multiple email addresses. The composite primary key 
  \texttt{(Coach\_ID,\ Email\_Address)} prevents duplicate email addresses 
  for the same coach.
\end{itemize}

\subsubsection{Training\_Session (Subclass of Reservation)}\label{training_session-training-course-reservation-subclass}

  \texttt{Training\_Session(Reservation\_ID\ PK,\\
  \hspace*{2em} Coach\_ID\ FK\ →\ Coach(Coach\_ID),\\
  \hspace*{2em} Max\_Capacity,\\
  \hspace*{2em} FK\ →\ Reservation(Reservation\_ID))}

  \begin{itemize}
  
  \item
    \texttt{Reservation\_ID} is both the primary key and foreign key,
    indicating that each course is a specific time period and facility
    occupancy;
  \item
    \texttt{Coach\_ID} records the teaching coach, implementing the 1:N
    relationship \texttt{Coach\ teaches\ Training\_Session} in EERD;
  \item
    \texttt{Max\_Capacity} represents the course capacity, providing
    basic data for enrollment control.
  \end{itemize}

\subsubsection{Session\_Enrollment (Week Entity)}\label{session_enrollment-course-enrollment-nm}


  \texttt{Session\_Enrollment(Reservation\_ID\ PK,\\
  \hspace*{2em} Member\_ID\ PK,\\
  \hspace*{2em} FK\ →\ Training\_Session(Reservation\_ID),\\
  \hspace*{2em} FK\ →\ Member(Member\_ID))}

  \begin{itemize}
  
  \item
    Converted from the N:M relationship between \texttt{Member} and
    \texttt{Training\_Session};
  \item
    The composite primary key \texttt{(Reservation\_ID,\ Member\_ID)}
    prevents the same member from repeatedly enrolling in the same
    course;
  \item
    Provides data foundation for subsequent statistics on course
    participation and member training frequency.
  \end{itemize}

\subsection{Maintenance Entity Relational
Schema}\label{maintenance-entity-relational-schema}

\texttt{Maintenance} is an independent strong entity used to record
maintenance plans and execution processes for equipment or facilities.
Its primary key is \texttt{Maintenance\_ID}, connected to
\texttt{Facility} and \texttt{Equipment} respectively through the
\texttt{requires} relationship with cardinality N:1. The conversion to
relational schema is as follows:

  \noindent \texttt{Maintenance(Maintenance\_ID\ PK,\\
  \hspace*{2em} Scheduled\_Date,\\
  \hspace*{2em} Completion\_Date,\\
  \hspace*{2em} Status,\\
  \hspace*{2em} Description,\\
  \hspace*{2em} Facility\_ID\ FK\ →\ Facility(Facility\_ID),\\
  \hspace*{2em} Equipment\_ID\ FK\ →\ Equipment(Equipment\_ID))}


\begin{itemize}

\item
  \texttt{Maintenance\_ID} is the unique identifier for maintenance
  records, no longer reusing \texttt{Reservation\_ID}, highlighting the
  independent business semantics of maintenance records;
\item
  \texttt{Scheduled\_Date} and \texttt{Completion\_Date} record the
  planned maintenance date and actual completion date respectively,
  which can be used to analyze maintenance delays;
\item
  \texttt{Status} reflects the maintenance task status, such as
  SCHEDULED / IN\_PROGRESS / COMPLETED;
\item
  Both \texttt{Facility\_ID} and \texttt{Equipment\_ID} can be null, but
  must satisfy the business rule: each maintenance record must
  be associated with either a facility or equipment, but not both null,
  and usually not both non-null. This can be implemented through CHECK
  constraints or application-level logic;
\item
  Maintenance records are not structurally strongly bound to
  Reservation, but selectively create corresponding Reservations for
  maintenance periods through upper-level logic when needed, to avoid
  continuing to open facilities for booking during maintenance time.
\end{itemize}


\subsection{SQL Definition Examples for Key
Relationships}\label{sql-definition-examples-for-key-relationships}

To highlight the implementation of primary keys, foreign keys, and key
constraints, this subsection provides MySQL DDL code snippets for
several representative relational schemas.

\subsubsection{Facility and
Reservation}\label{facility-and-reservation}

\begin{sql}
CREATE TABLE Facility (
	Facility_ID     INT UNSIGNED PRIMARY KEY,
	Facility_Name   VARCHAR(100) NOT NULL,
	Type            VARCHAR(30)  NOT NULL,
	Status          ENUM('Available', 'Under_Maintenance', 'Unavailable') NOT NULL,
	Capacity        INT UNSIGNED NOT NULL CHECK (Capacity > 0),
	Building        VARCHAR(50)  NOT NULL,
	Floor           TINYINT      NOT NULL,
	Room_Number     VARCHAR(10)  NOT NULL
);

CREATE TABLE Reservation (
	Reservation_ID   INT UNSIGNED PRIMARY KEY,
	Facility_ID      INT UNSIGNED NOT NULL,
	Reservation_Date DATE        NOT NULL,
	Start_Time       TIME        NOT NULL,
	End_Time         TIME        NOT NULL,
	CONSTRAINT fk_reservation_facility
		FOREIGN KEY (Facility_ID)
		REFERENCES Facility(Facility_ID),
	CONSTRAINT chk_reservation_time
		CHECK (Start_Time < End_Time)
);
\end{sql}

\subsubsection{Booking: Preventing Duplicate Claims for the Same
Time
Slot}\label{booking-preventing-duplicate-claims-for-the-same-time-slot}

\begin{sql}
CREATE TABLE Booking (
    Reservation_ID  INT UNSIGNED PRIMARY KEY,
    Member_ID       INT UNSIGNED NOT NULL,
    Booking_Status  ENUM('Pending', 'Confirmed', 'Cancelled') NOT NULL,
    CONSTRAINT fk_booking_reservation
        FOREIGN KEY (Reservation_ID)
        REFERENCES Reservation(Reservation_ID),
    CONSTRAINT fk_booking_member
        FOREIGN KEY (Member_ID)
        REFERENCES Member(Member_ID)
);
\end{sql}

\subsubsection{Reservation\_Equipments: Equipment Usage
Record}\label{reservation_equipments-equipment-usage-record}

\begin{sql}
CREATE TABLE Reservation_Equipments (
    Reservation_ID  INT UNSIGNED NOT NULL,
    Equipment_ID    INT UNSIGNED NOT NULL,
    Quantity        INT UNSIGNED NOT NULL CHECK (Quantity > 0),
    PRIMARY KEY (Reservation_ID, Equipment_ID),
    CONSTRAINT fk_re_reservation
        FOREIGN KEY (Reservation_ID)
        REFERENCES Reservation(Reservation_ID),
    CONSTRAINT fk_re_equipment
        FOREIGN KEY (Equipment_ID)
        REFERENCES Equipment(Equipment_ID)
);
\end{sql}

\subsubsection{Training\_Session and
Session\_Enrollment}\label{training_session-and-session_enrollment}

\begin{sql}
CREATE TABLE Training_Session (
    Reservation_ID  INT UNSIGNED PRIMARY KEY,
    Coach_ID        INT UNSIGNED NOT NULL,
    Max_Capacity    INT UNSIGNED NOT NULL CHECK (Max_Capacity > 0),
    CONSTRAINT fk_ts_reservation
        FOREIGN KEY (Reservation_ID)
        REFERENCES Reservation(Reservation_ID),
    CONSTRAINT fk_ts_coach
        FOREIGN KEY (Coach_ID)
        REFERENCES Coach(Coach_ID)
);

CREATE TABLE Session_Enrollment (
    Reservation_ID  INT UNSIGNED NOT NULL,
    Member_ID       INT UNSIGNED NOT NULL,
    PRIMARY KEY (Reservation_ID, Member_ID),
    CONSTRAINT fk_se_session
        FOREIGN KEY (Reservation_ID)
        REFERENCES Training_Session(Reservation_ID),
    CONSTRAINT fk_se_member
        FOREIGN KEY (Member_ID)
        REFERENCES Member(Member_ID)
);
\end{sql}

\subsubsection{Maintenance: Independent Maintenance
Record}\label{maintenance-independent-maintenance-record}

\begin{sql}
CREATE TABLE Maintenance (
	Maintenance_ID   INT UNSIGNED PRIMARY KEY,
	Scheduled_Date   DATE        NOT NULL,
	Completion_Date  DATE        NULL,
	Status           ENUM('Scheduled', 'In_Progress', 'Completed', 'Cancelled') NOT NULL,
	Description      TEXT        NULL,
	Facility_ID      INT UNSIGNED NULL,
	Equipment_ID     INT UNSIGNED NULL,
	-- Note: Each maintenance record represents one equipment instance or one facility.
	-- Multiple equipment items require multiple maintenance records.
	CONSTRAINT fk_maintenance_facility
		FOREIGN KEY (Facility_ID)
		REFERENCES Facility(Facility_ID),
	CONSTRAINT fk_maintenance_equipment
		FOREIGN KEY (Equipment_ID)
		REFERENCES Equipment(Equipment_ID),
	CONSTRAINT chk_maintenance_target
		CHECK (
			(Facility_ID IS NOT NULL AND Equipment_ID IS NULL)
			OR
			(Facility_ID IS NULL AND Equipment_ID IS NOT NULL)
		)
);
\end{sql}

Through the above key DDL code, the implementation of primary keys,
foreign keys, and CHECK constraints in relational schemas can be more
intuitively demonstrated, laying the foundation for the next section on
physical design analysis.

\section{Physical Design and Rationality Analysis}\label{physical-design-and-rationality-analysis-task-4}

After completing the logical schema design, the physical design phase
needs to make specific data type selections, constraint and indexing
strategies for the target DBMS (MySQL in this case), and demonstrate the
rationality of the design from perspectives such as "meeting business
requirements" and "effectively preventing booking conflicts".

This section analyzes from four aspects: data types, constraints,
indexes, and key business rules.

\subsection{Data Type and Constraint
Selection}\label{data-type-and-constraint-selection}

\subsubsection{Identifiers and Primary Key
Fields}\label{identifiers-and-primary-key-fields}

The identifiers of each entity (such as \texttt{Member\_ID}, 
\texttt{Facility\_ID}, \texttt{Equipment\_ID}, \texttt{Reservation\_ID},
\texttt{Application\_ID}, \texttt{Maintenance\_ID}, etc.) uniformly
adopt:

\begin{sql}
INT UNSIGNED
\end{sql}

Reasons:

\begin{enumerate}
\def\labelenumi{\arabic{enumi}.}

\item
  Facilitates the use of \texttt{AUTO\_INCREMENT} for automatic
  numbering, simplifying insertion logic;
\item
  Unsigned integers provide a larger range of positive values, meeting
  the primary key growth requirements under long-term use;
\item
  Works well with InnoDB's clustered index mechanism, providing good
  insertion and lookup performance.
\end{enumerate}

The primary keys of subclass tables (such as \texttt{Student},
\texttt{Staff}, \texttt{External\_Visitor}, \texttt{Training\_Session},
\texttt{Booking}) all reuse the superclass primary key, physically also
using \texttt{INT\ UNSIGNED}, ensuring simple and clear foreign key
references.

\subsubsection{Text and Descriptive
Fields}\label{text-and-descriptive-fields}

\begin{itemize}
\item
  Person names, facility names, equipment names, etc., use the
  \texttt{VARCHAR} type:

\begin{sql}
First_Name      VARCHAR(50),
Last_Name       VARCHAR(50),
Facility_Name   VARCHAR(100),
Equipment_Name  VARCHAR(100)  
\end{sql}

  Supports mixed Chinese-English text, with lengths balancing space and
  actual usage scenarios.
\item
  Descriptive fields such as \texttt{Description},
  \texttt{Reject\_Reason} use the \texttt{TEXT} type to accommodate
  longer text, suitable for complex maintenance descriptions or
  rejection reasons.
\end{itemize}

\subsubsection{Date and Time Fields}\label{date-and-time-fields}

\begin{itemize}

\item
  Date: \texttt{Reservation\_Date}, \texttt{Application\_Date},
  \texttt{Approval\_Date}, \texttt{Scheduled\_Date},
  \texttt{Completion\_Date}, etc., use the \texttt{DATE} type;
\item
  Time: \texttt{Start\_Time}, \texttt{End\_Time} use the \texttt{TIME}
  type.
\end{itemize}

Separating date and time has the following advantages:

\begin{enumerate}
\def\labelenumi{\arabic{enumi}.}

\item
  Facilitates statistics by date dimension, such as booking volume and
  maintenance volume (e.g., "facility utilization rate for a certain
  week");
\item
  Time conflict calculations are simple, focusing on start and end times
  within the same day, more intuitive than operating complete
  timestamps;
\item
  Reserves space for future expansion to cross-day bookings (can be
  implemented through additional flags or date ranges).
\end{enumerate}

\subsubsection{Status Fields and ENUM
Usage}\label{status-fields-and-enum-usage}

For status fields with limited value ranges and clear business semantics
(such as \\ \texttt{Booking\_Status}, \texttt{Membership\_Status},
\texttt{Maintenance.Status}, \texttt{Facility.Status}, etc.), MySQL's
\texttt{ENUM} type is uniformly adopted, for example:

\begin{sql}
Membership_Status ENUM('Active', 'Inactive', 'Pending_Approval') NOT NULL;  
\end{sql}

Benefits include:

\begin{enumerate}
\def\labelenumi{\arabic{enumi}.}

\item
  Embeds business rules at the data type level, avoiding spelling errors
  or illegal status values;
\item
  Query conditions are concise, improving SQL readability;
\item
  Easily corresponds to frontend enumeration constants, reducing
  front-end and back-end coordination costs.
\end{enumerate}

\subsubsection{Numeric Fields and Check
Constraints}\label{numeric-fields-and-check-constraints}

Capacity and quantity fields (such as \texttt{Capacity},
\texttt{Total\_Quantity}, \texttt{Max\_Capacity}, \texttt{Quantity})
uniformly use \texttt{INT\ UNSIGNED} and are given
\texttt{CHECK\ \textgreater{}\ 0} constraints:

\begin{sql}
Capacity     INT UNSIGNED NOT NULL CHECK (Capacity > 0);
Total_Quantity INT UNSIGNED NOT NULL CHECK (Total_Quantity > 0);
\end{sql}

These constraints ensure domain integrity, preventing
unreasonable situations such as negative numbers or zero capacity.

The derived attribute \texttt{Available\_Quantity} is not stored as a static field but dynamically calculated \textbf{for a specific time slot} through queries. For example:

\begin{sql}
SELECT  e.Equipment_ID,
      e.Equipment_Name,
      e.Total_Quantity
      - IFNULL(SUM(re.Quantity), 0)  -- subtract reserved quantity
      - IFNULL(COUNT(DISTINCT m.Maintenance_ID), 0)  -- subtract maintenance count
      AS Available_Quantity
FROM Equipment e
LEFT JOIN Reservation_Equipments re
       ON e.Equipment_ID = re.Equipment_ID
      -- Filter active reservations overlapping with the REQUESTED TIME SLOT
LEFT JOIN Maintenance m
       ON e.Equipment_ID = m.Equipment_ID
      AND m.Status IN ('Scheduled', 'In_Progress')  -- only active maintenance during the REQUESTED TIME SLOT
WHERE ...
GROUP BY e.Equipment_ID, e.Equipment_Name, e.Total_Quantity;
\end{sql}

This approach avoids distortion of available quantities due to
unsynchronized updates, while still meeting performance requirements
under appropriate indexing.

\subsection{Index Strategy and Query
Performance}\label{index-strategy-and-query-performance}

Based on typical operational requirements in Scenario 1 (such as "view
booking status for a facility on a specific day", "query courses by
coach", "query historical bookings by member", etc.), this design
focuses on establishing auxiliary indexes for the following fields:

\begin{enumerate}
\def\labelenumi{\arabic{enumi}.}
\item
  \textbf{Reservation Level}:

  \begin{itemize}
  
  \item
    \texttt{(Facility\_ID,\ Reservation\_Date,\ Start\_Time)} composite
    unique index, used for both conflict detection and high-frequency
    queries;
  \end{itemize}
\item
  \textbf{Booking Level}:

  \begin{itemize}
  
  \item
    Secondary index on \texttt{Booking(Member\_ID)} to accelerate
    querying all booking records by member;
  \end{itemize}
\item
  \textbf{Training\_Session / Session\_Enrollment Level}:

  \begin{itemize}
  
  \item
    \texttt{Training\_Session(Coach\_ID)}: for listing all courses by
    coach;
  \item
    \texttt{Session\_Enrollment(Member\_ID)}: for querying course
    participation records by member;
  \end{itemize}
\item
  \textbf{Maintenance Level}:

  \begin{itemize}
  
  \item
    \texttt{Maintenance(Facility\_ID,\ Scheduled\_Date)}: for filtering
    maintenance tasks by facility and date;
  \item
    \texttt{Maintenance(Equipment\_ID,\ Scheduled\_Date)}: for querying
    maintenance plans by equipment and date.
  \end{itemize}
\end{enumerate}

Through the above index settings, the most common business query paths
can be significantly optimized, allowing the system to maintain good
response times under high concurrent bookings and frequent queries.

\subsection{Multi-layer Mechanism for Preventing Double
Booking}\label{multi-layer-mechanism-for-preventing-double-booking}

Preventing double booking is one of the core objectives of this system
design. This design adopts a three-layer linkage of "structural
constraints + transaction control + business logic" at the database
level to ensure that facilities and equipment are not repeatedly
allocated during the same time period.

\subsubsection{Structural Layer: Unique Constraint on
Reservation}\label{structural-layer-unique-constraint-on-reservation}

Add the following unique constraint to the \texttt{Reservation} table:

\begin{sql}
ALTER TABLE Reservation
ADD CONSTRAINT uq_facility_timeslot
    UNIQUE (Facility_ID, Reservation_Date, Start_Time); 
\end{sql}

This constraint serves as a \textbf{first line of defense} at the database structural level. It prevents distinct records from starting at the exact same moment for the same facility. While it does not structurally prevent overlapping time intervals (e.g., 10:00--12:00 vs. 11:00--13:00), it ensures that no two reservations can have identical start times, simplifying the application-level conflict checks described below.

In simple modeling, we use "start time" as the time granularity. To
further prevent any time interval overlap (not just the same start
time), time interval overlap check logic can be added at the
application layer, executed within a transaction:

\begin{sql}
SELECT COUNT(*) AS cnt
FROM Reservation
WHERE Facility_ID = @facility_id
  AND Reservation_Date = @reservation_date
  AND NOT (@new_end_time <= Start_Time OR @new_start_time >= End_Time);
\end{sql}

If \texttt{cnt\ \textgreater{}\ 0}, it indicates there is a time
interval overlap, and the creation of the new Reservation should be
rejected.

\subsubsection{Structural Layer: 1:1 Mapping of
Booking}\label{structural-layer-11-mapping-of-booking}

The primary key of \texttt{Booking} is \texttt{Reservation\_ID}:

\begin{sql}
PRIMARY KEY (Reservation_ID)
FOREIGN KEY (Reservation_ID) REFERENCES Reservation(Reservation_ID)
\end{sql}

This brings two direct effects:

\begin{enumerate}
\def\labelenumi{\arabic{enumi}.}

\item
  The same time period (same Reservation) can be "bound" by at most
  one Booking instance, structurally avoiding the situation of "the
  same time period being booked by multiple members simultaneously";
\item
  Other subclasses (such as \texttt{Training\_Session}) use the same
  primary key pushdown method, preventing the same Reservation from
  being reused by multiple subclasses with different semantics.
\end{enumerate}

\subsubsection{Structural Layer: Session\_Enrollment Prevents Duplicate
Enrollment}\label{structural-layer-session_enrollment-prevents-duplicate-enrollment}

\texttt{Session\_Enrollment} uses the composite primary key
\texttt{(Reservation\_ID,\ Member\_ID)} to prevent members from
repeatedly enrolling in the same course:

\begin{sql}
PRIMARY KEY (Reservation_ID, Member_ID)
\end{sql}

If a record with the same \texttt{(Reservation\_ID,\ Member\_ID)} is
inserted, it violates the primary key constraint. Although this
constraint does not belong to "facility double booking", it ensures
the rationality of course enrollment business.

\subsubsection{Business Logic Layer: Equipment Availability and
Concurrency
Control}\label{business-logic-layer-equipment-availability-and-concurrency-control}

The "hidden conflict" of equipment comes from over-allocation of the
same type of equipment during the same time period. Using
\texttt{Reservation\_Equipments} and \texttt{Equipment.Total\_Quantity},
when inserting or updating equipment allocation, the following check can
be executed within a transaction:

\begin{sql}
START TRANSACTION;

SELECT  e.Total_Quantity - IFNULL(SUM(re.Quantity), 0) AS Available
FROM Equipment e
LEFT JOIN Reservation_Equipments re
       ON e.Equipment_ID = re.Equipment_ID
      AND re.Reservation_ID IN (
          -- Find Reservation_IDs that temporally overlap with the current request
      )
WHERE e.Equipment_ID = @equipment_id
FOR UPDATE;

-- If Available < @request_quantity, then ROLLBACK and return "insufficient equipment" error
-- Otherwise, execute insert/update Reservation_Equipments and COMMIT
\end{sql}

By using \texttt{FOR\ UPDATE} to lock relevant rows, even in high
concurrency environments, different transactions can be prevented from
simultaneously reading "un-updated" available quantities, thereby
avoiding over-allocation issues.

\subsection{Rationality of Independent Maintenance
Modeling}\label{rationality-of-independent-maintenance-modeling}

Compared to modeling Maintenance as a Reservation subclass, this
revision maintains consistency with the EERD description and has the
following advantages:

\begin{enumerate}
\def\labelenumi{\arabic{enumi}.}
\item
  \textbf{Clear Semantics and Separation of Concerns}\\ Maintenance
  focuses on the maintenance process itself (planned date, completion
  date, status, description, etc.), while Reservation focuses on time
  periods and facility occupancy. The logical division of labor between
  the two is clear.
\item
  \textbf{Conforms to EERD Description Structure}\\ Using
  \texttt{Maintenance\_ID} as the primary key and establishing N:1
  relationships with corresponding entities through
  \texttt{Facility\_ID} / \texttt{Equipment\_ID} fully corresponds to
  the \texttt{requires} relationship and cardinality settings in the
  specification document.
\item
  \textbf{Good Extensibility}\\ If attributes such as maintenance cost,
  third-party service providers, or responsible persons need to be added
  later, only the Maintenance table needs to be extended without
  affecting the Reservation structure on the timeline.
\item
  \textbf{Transaction-Level Consistency}\\ While logically separated, strict consistency is maintained via the transaction layer. Booking transactions explicitly lock and check the \texttt{Maintenance} table for overlaps before insertion, ensuring no facility is booked while under maintenance. This approach maintains the clean separation of concerns while ensuring operational integrity through rigorous transaction logic.
\item
  \textbf{Easy Auditing and Statistical Analysis}\\ Maintenance records
  exist independently, suitable for direct use in statistical reports,
  such as "annual maintenance count", "average maintenance
  duration", "facility maintenance frequency ranking", etc.
  Decoupling from booking records is more conducive to separate
  analysis.
\end{enumerate}

\newpage
\section{Access Control}\label{task-5-access-control}

\subsection{User Roles and
Responsibilities}\label{user-roles-and-responsibilities}

In our implementation, database roles (DBA, MANAGER, BOOKING\_OFFICER, COACH,
MEMBER) are mapped to application users stored in the \texttt{Staff},
\texttt{Member}, \texttt{Student}, and \texttt{External\_Visitor} tables.
Role-based access control (RBAC) is widely adopted for assigning permissions
based on organizational roles. Modern studies, such as
\citet{frank2013roleminingprobabilisticmodels}, further demonstrate how RBAC
structures can be systematically derived and optimized from user–permission
relationships.A staff member can, for example,
be marked as "manager" or "front desk" in the application; at login,
the corresponding database role is enabled for that user.

Below we describe each logical role and its responsibilities in the
sports complex system.

\subsubsection{System Administrator
(DBA)}\label{system-administrator-dba}

The System Administrator is the technical person who manages the
database platform.

Main responsibilities:

\begin{itemize}

\item
  Create and disable database accounts, assign roles, and reset
  passwords.
\item
  Manage the physical database objects (tables, indexes, views),
  backups, and recovery.
\item
  Perform data correction only when formally requested and authorised by
  management.
\end{itemize}

From a security perspective, the DBA can run any SQL command,
but in normal operations the DBA focuses on availability, performance,
and integrity rather than on day‑to‑day business data entry or
approvals.

\subsubsection{Sports Complex Manager}\label{sports-complex-manager}

The Sports Complex Manager is responsible for the overall operation of
the sports complex and for higher‑level business decisions.

Main responsibilities:

\begin{itemize}

\item
  Maintain core reference data such as \texttt{Facility} and
  \texttt{Equipment} (e.g.~names, capacities, availability status).
\item
  Define and maintain \texttt{Training\_Session} records, including
  assigned coach, schedule, capacity, and prerequisites.
\item
  Review maintenance history and approve completion of maintenance jobs.
\item
  Review conflicts between reservations and ensure that important events
  (e.g.~tournaments) are scheduled correctly.
\item
  Approve or reject \texttt{Visitor\_Application} records for
  external visitors.
\end{itemize}

The Manager therefore needs broad read access over the operational data
and the ability to update key business objects. However, the Manager
generally does not hard‑delete business records (such as
\texttt{Member} or \texttt{Facility}) so that history is preserved;
instead, status fields (e.g.~\texttt{Active/Inactive}) are used.

\subsubsection{Front Desk / Booking
Officer}\label{front-desk-booking-officer}

Front desk staff handle most of the daily interactions with members and
visitors.

Main responsibilities:

\begin{itemize}

\item
  Register new members and assist them in updating their profiles.
\item
  Help members make, modify, and cancel bookings and session enrolments
  (walk‑in or phone requests).
\item
  Record maintenance requests on behalf of members or staff, linking
  them to the relevant facility/equipment.
\item
  Answer enquiries about facility availability, upcoming sessions, and
  existing bookings.
\end{itemize}

Operationally, the Booking Officer needs read/write access to day‑to‑day
transaction tables such as \texttt{Booking},
\texttt{Session\_Enrollment}, and \texttt{Maintenance}, but cannot
change high‑level configuration such as facility capacity or approve
external visitor applications. That separation reduces the risk of
accidental or unauthorised changes to critical data.

\subsubsection{Coach}\label{coach}

Coaches are staff members responsible for delivering training sessions.

Main responsibilities:

\begin{itemize}

\item
  View their own \texttt{Training\_Session} schedule and session
  details.
\item
  View the list of enrolled members for their sessions
  (\texttt{Session\_Enrollment}).
\item
  Record attendance, remarks, or simple outcome indicators for
  participants in their own sessions.
\end{itemize}

Coaches do not create or delete training sessions. They also
cannot see all member information; their access is restricted to members
who are enrolled in sessions that they teach.

\subsubsection{Member (Student / Staff / External
Visitor)}\label{member-student-staff-external-visitor}

A Member is any person allowed to use the sports complex: internal
students, internal staff, and approved external visitors. All three
subtypes share the same high‑level behaviour with minor differences in
attributes.

Main responsibilities:

\begin{itemize}

\item
  Maintain their own personal details (contact information, emergency
  contact, etc.).
\item
  Browse the list of facilities, equipment, and training sessions.
\item
  Make, modify, and cancel their own bookings, subject to business rules
  (time limits, clashes, etc.).
\item
  Enrol in and withdraw from training sessions, again limited to their
  own records.
\end{itemize}

Members never directly modify data that belongs to other members or to
the sports complex configuration. Access is restricted to "own" data
(for example, own bookings and enrolments) using row‑level security.

\subsection{Access Types for Each
Table}\label{access-types-for-each-table}

We classify access as:

\begin{itemize}

\item
  \textbf{R} -- Read (SELECT)
\item
  \textbf{I} -- Insert (CREATE new row)
\item
  \textbf{U} -- Update (modify existing row)
\item
  \textbf{D} -- Delete
\end{itemize}

A star \texttt{*} indicates that the access is limited to the
user's own rows, enforced by predicates on \texttt{Member\_ID} (or the
equivalent foreign key). A double star \texttt{**} indicates access
restricted to rows related to the coach's own sessions (for
example, enrolments in sessions they teach).

The correctness of such permission assignments is essential, and recent work has
shown that access-control policies can be formally analyzed and verified to
avoid misconfigurations \citep{gouglidis2023modelcheckingaccesscontrol}.

\subsubsection{Identity and Contact
Tables}\label{identity-and-contact-tables}

These tables store member and coach identities and their contact
details.

\begin{table}[H]
\centering
\begin{tabular}{@{}lccccc@{}}
\toprule
Table & DBA & Manager & Booking Officer & Coach & Member \\
\midrule
Member & R/I/U/D & R/I/U & R/I/U & R & R/U* \\
Student & R/I/U/D & R/I/U & R/I/U & R & R/U* \\
Staff & R/I/U/D & R/I/U & R/I/U & R & R/U* \\
External\_Visitor & R/I/U/D & R/I/U & R/I/U & R & R/U* \\
Coach & R/I/U/D & R/I/U & R & R/I/U* & R \\
Member\_Phone & R/I/U/D & R/I/U & R/I/U & R & R/I/U* \\
Member\_Email & R/I/U/D & R/I/U & R/I/U & R & R/I/U* \\
Coach\_Phone & R/I/U/D & R/I/U & R & R/I/U* & R \\
Coach\_Email & R/I/U/D & R/I/U & R & R/I/U* & R \\
\bottomrule
\end{tabular}
\end{table}

Rationale:

\begin{itemize}

\item
  The DBA has full control for administrative and recovery purposes.
\item
  The Manager can maintain staff and member data but avoids hard deletes
  in normal practice.
\item
  Booking Officers need to create and update member
  records when assisting people at the front desk. They have unrestricted
  read, insert, and update access to member-related tables because their
  primary job is to help \emph{any} member with registration, profile
  updates, and booking management. Application-level logic (e.g., audit logs,
  transaction context) ensures that these operations are performed only in
  legitimate service scenarios.
\item
  Each Member can read and update only their own profile and
  contact details, enforced via predicates such as
  \texttt{Member\_ID\ =\ CURRENT\_USER} (or an equivalent context
  mechanism).
\item
  Coaches maintain only their own identity and contact details in these
  tables. When they need to contact participants, the application
  exposes filtered views that show minimal member contact information
  only for participants enrolled in their sessions, not for all members
  in the system.
\end{itemize}

Row‑level security for the \texttt{*} and \texttt{**} restrictions will
be implemented using views or fine‑grained access predicates (see
example below).

\subsubsection{Facility, Equipment, and Maintenance
Tables}\label{facility-equipment-and-maintenance-tables}

\begin{table}[H]
\centering
\begin{tabular}{@{}lccccc@{}}
\toprule
Table & DBA & Manager & Booking Officer & Coach & Member \\
\midrule
Facility & R/I/U/D & R/I/U & R & R & R \\
Equipment & R/I/U/D & R/I/U & R & R & R \\
Maintenance & R/I/U/D & R/I/U & R/I/U & R & R \\
\bottomrule
\end{tabular}
\end{table}

Rationale:

\begin{itemize}

\item
  \texttt{Facility} and \texttt{Equipment} define the core resources of
  the sports complex. They are therefore maintained by the Manager;
  other roles have read‑only access.
\item
  \texttt{Maintenance} records may be created either by the Manager or
  by the Booking Officer when a member reports a problem. In practice,
  application logic or column‑level controls ensure that only
  the manager can set a job to completed
  (e.g.~\texttt{Status\ =\ \textquotesingle{}Completed\textquotesingle{}}
  and \texttt{Completion\_Date}), while Booking Officers can update
  descriptive fields such as problem description or notes.
\item
  Members and Coaches can read maintenance information to understand why
  a resource is unavailable, but cannot create or alter maintenance
  records.
\end{itemize}

\subsubsection{Reservation, Booking, and Training
Tables}\label{reservation-booking-and-training-tables}

The conceptual model treats \texttt{Reservation} as a supertype of
\texttt{Booking} and \texttt{Training\_Session}. In practice, end users
mainly interact with higher‑level entities through application views;
direct access to the base \texttt{Reservation} table is mostly required
for administrative purposes.

\begin{table}[H]
\centering
\begin{tabular}{@{}lccccc@{}}
\toprule
Table & DBA & Manager & Booking Officer & Coach & Member \\
\midrule
Reservation & R/I/U/D & R/U & R/I/U & R** & R* \\
Booking & R/I/U/D & R/U & R/I/U & R** & R/I/U* \\
Training\_Session & R/I/U/D & R/I/U/D & R & R/U** & R \\
Reservation\_Equipments & R/I/U/D & R/I/U & R/I/U & R** & R* \\
Session\_Enrollment & R/I/U/D & R/U & R/I/U & R/U** & R/I/U* \\
\bottomrule
\end{tabular}
\end{table}

Rationale:

\begin{itemize}
\item
  \textbf{Member}:

  \begin{itemize}
  
  \item
    Can create, update, and cancel their own \texttt{Booking} and
    \texttt{Session\_Enrollment} records (\texttt{*}).
  \item
    Has read‑only access to their underlying \texttt{Reservation}
    records, normally via views rather than direct table access.
  \item
    Can view the equipment associated with their own reservations via
    application views over \texttt{Reservation\_Equipments}, but cannot
    directly insert, update, or delete rows in that table.
  \end{itemize}
\item
  \textbf{Booking Officer}:

  \begin{itemize}
  
  \item
    Handles creation, modification, and cancellation of \texttt{Booking}
    and \texttt{Session\_Enrollment} for any member.
  \item
    Can also create or adjust \texttt{Reservation} and
    \texttt{Reservation\_Equipments} entries to support bookings and
    training sessions.
  \end{itemize}
\item
  \textbf{Manager}:

  \begin{itemize}
  
  \item
    Oversees overall scheduling and therefore needs read and update
    rights on \texttt{Reservation}, \texttt{Booking}, and
    \texttt{Session\_Enrollment} to resolve conflicts or correct errors.
  \item
    Has full control over \texttt{Training\_Session} (create, update,
    and, in exceptional cases, delete), including assigning coaches and
    setting capacities.
  \end{itemize}
\item
  \textbf{Coach}:

  \begin{itemize}
  
  \item
    Can read all relevant reservation and booking details for
    their own sessions only (\texttt{**}).
  \item
    Can update their own \texttt{Training\_Session} details where
    permitted (e.g.~remarks, minor time adjustments approved by the
    Manager) and record attendance or notes in
    \texttt{Session\_Enrollment} for participants in their sessions.
  \end{itemize}
\end{itemize}

In an actual DBMS implementation, most of the member‑facing operations
would be performed through views such as \texttt{Member\_Booking} and
\texttt{Member\_Session\_Enrollment}, which already filter rows by
\texttt{Member\_ID} and prevent access to other members' data.

\subsubsection{Visitor Application
Management}\label{visitor-application-management}

\begin{table}[H]
\centering
\begin{tabular}{@{}lccccc@{}}
\toprule
Table & DBA & Manager & Booking Officer & Coach & Member \\
\midrule
Visitor\_Application & R/I/U/D & R/I/U & R/I & R & -- \\
Visitor\_Application\_Phone & R/I/U/D & R/I/U & R/I & R & -- \\
Visitor\_Application\_Email & R/I/U/D & R/I/U & R/I & R & -- \\
\bottomrule
\end{tabular}
\end{table}

Rationale:

\begin{itemize}

\item
  Prospective visitors (not yet members) submit applications, typically
  through the front desk or an online form. Booking Officers can create
  and insert \texttt{Visitor\_Application} records on behalf of walk-in
  applicants.
\item
  The Manager reviews applications, updates their status
  (e.g.~\texttt{Pending}, \texttt{Approved}, \texttt{Rejected}), and
  records approval details. Upon approval, the Manager creates
  corresponding \texttt{Member} and \texttt{External\_Visitor} records
  and updates the \texttt{Created\_Member\_ID} field for traceability.
\item
  Members do not have access to \texttt{Visitor\_Application} tables
  because these records represent people who are not yet in the system.
  Once an application is approved and a \texttt{Member} record is created,
  the applicant gains member privileges.
\item
  Hard deletes are avoided in normal operations to keep an audit trail
  of all applications.
\end{itemize}

\subsection{Example SQL Implementation (Roles, Grants, and
Views)}\label{example-sql-implementation-roles-grants-and-views}

Below is a simplified example of how this access control model could be
implemented in SQL. Exact syntax may vary between DBMSs.

\begin{sql}
-- Create roles
CREATE ROLE dba_role;              -- mapped to the real DBA account
CREATE ROLE manager_role;          -- sports complex manager
CREATE ROLE booking_officer_role;  -- front desk staff
CREATE ROLE coach_role;            -- coaches
CREATE ROLE member_role;           -- regular members

-- Example: grant privileges on Booking
GRANT SELECT, INSERT, UPDATE, DELETE ON Booking TO dba_role;
GRANT SELECT, UPDATE ON Booking TO manager_role;
GRANT SELECT, INSERT, UPDATE ON Booking TO booking_officer_role;

-- Member should only see and modify their own bookings.
-- We create a view that filters on MEMBER_ID and only expose the view to the member_role.

CREATE VIEW Member_Booking AS
SELECT b.*
FROM Booking b
JOIN Member m ON b.Member_ID = m.Member_ID
WHERE m.Member_ID = CURRENT_USER;  -- or another context function depending on DBMS

GRANT SELECT, INSERT, UPDATE ON Member_Booking TO member_role;
REVOKE ALL ON Booking FROM member_role;

-- Similarly, we can define a view for coaches to see enrolments only in their own sessions.

CREATE VIEW Coach_Session_Enrollment AS
SELECT se.*
FROM Session_Enrollment se
JOIN Training_Session ts ON se.Reservation_ID = ts.Reservation_ID
JOIN Coach c ON ts.Coach_ID = c.Coach_ID
WHERE c.Coach_ID = CURRENT_USER;  -- or another context function depending on DBMS

GRANT SELECT, UPDATE ON Coach_Session_Enrollment TO coach_role;
REVOKE ALL ON Session_Enrollment FROM coach_role;
\end{sql}

In a real deployment we would also use schemas, additional views, or
fine‑grained access control mechanisms (depending on the DBMS) to
enforce the \texttt{*} and \texttt{**} row‑level restrictions
consistently across all tables.

\subsection{Security Principles
Demonstrated}\label{security-principles-demonstrated}

This design implements several core data security principles:

\begin{enumerate}
\def\labelenumi{\arabic{enumi}.}
\item
  \textbf{Least privilege}\\ Each role receives only the permissions
  necessary for its duties. For example, members can modify only their
  own bookings and profiles; coaches can see only the enrolments for
  sessions they teach; booking officers cannot change facility capacity
  or approve applications.
\item
  \textbf{Separation of duties}\\ Sensitive actions such as approving
  external visitor registrations or marking maintenance as completed are
  reserved for the Manager. Front desk staff can collect and record data
  but cannot finalise approvals or override important business rules.
\item
  \textbf{Data privacy}\\ Personal data is protected via row‑level
  restrictions and dedicated views. Members see only their own identity,
  contact details, bookings, and applications. Coaches see limited
  information about participants in their sessions, and not about all
  members in the system.
\item
  \textbf{Auditability and integrity}\\ Instead of deleting key business
  records, we generally mark them as cancelled, inactive, or completed
  using status attributes. Physical deletion is reserved for the DBA in
  exceptional cases (e.g.~test data or legal requirements). This keeps a
  reliable audit trail of all important operations (bookings,
  maintenance, applications, etc.).
\end{enumerate}

Overall, this role‑based access control (RBAC) design is consistent with
our EERD and supports the daily workflow of the sports complex while
providing a reasonable level of security, privacy, and maintainability.

\section{Database Integrity}\label{task-6-integrity-constraints}

This section explains how entity integrity, referential integrity, and
domain integrity are enforced in our logical design. We link each type
of integrity to the tables and attributes generated from the EERD.

\subsection{Entity Integrity}\label{entity-integrity}

Entity integrity ensures that every row in a table can be uniquely
identified and that primary key attributes are never \texttt{NULL}.

\subsubsection{Strong Entities and Primary
Keys}\label{strong-entities-and-primary-keys}

Each strong entity in the EERD becomes a base table with a single,
non-null primary key:

\begin{itemize}

\item
  \texttt{Member(Member\_ID,\ ...)} with
  \texttt{Member\_ID\ INT\ PRIMARY\ KEY\ NOT\ NULL}
\item
  \texttt{Coach(Coach\_ID,\ ...)} with
  \texttt{Coach\_ID\ INT\ PRIMARY\ KEY\ NOT\ NULL}
\item
  \texttt{Facility(Facility\_ID,\ ...)} with
  \texttt{Facility\_ID\ INT\ PRIMARY\ KEY\ NOT\ NULL}
\item
  \texttt{Equipment(Equipment\_ID,\ ...)} with
  \texttt{Equipment\_ID\ INT\ PRIMARY\ KEY\ NOT\ NULL}
\item
  \texttt{Reservation(Reservation\_ID,\ ...)} with
  \texttt{Reservation\_ID\ INT\ PRIMARY\ KEY\ NOT\ NULL}
\item
  \texttt{Maintenance(Maintenance\_ID,\ ...)} with
  \texttt{Maintenance\_ID\ INT\ PRIMARY\ KEY\ NOT\ NULL}
\item
  \texttt{Visitor\_Application(Application\_ID,\ ...)} with
  \texttt{Application\_ID\ INT\ PRIMARY\ KEY\ NOT\ NULL}
\end{itemize}

These primary keys uniquely identify each row and are defined as
\texttt{NOT\ NULL}, so no row can exist without an identifier.

\subsubsection{Supertype--Subtype Structure and Key
Inheritance}\label{supertypesubtype-structure-and-key-inheritance}

\texttt{Member} is a supertype with three disjoint subtypes:
\texttt{Student}, \texttt{Staff}, and \texttt{External\_Visitor}. Each
subtype table uses the same primary key as \texttt{Member} to preserve
the 1:1 relationship between a member and its subtype specialization:

\begin{sql}
CREATE TABLE Member (
    Member_ID INT PRIMARY KEY,
    Membership_Status VARCHAR(20) NOT NULL,
    -- other attributes...
);

CREATE TABLE Student (
    Member_ID  INT PRIMARY KEY,
    Student_ID VARCHAR(20) NOT NULL UNIQUE,
    -- other attributes...
    FOREIGN KEY (Member_ID) REFERENCES Member(Member_ID)
        ON DELETE CASCADE
);

CREATE TABLE Staff (
    Member_ID INT PRIMARY KEY,
    Staff_ID  VARCHAR(20) NOT NULL UNIQUE,
    -- other attributes...
    FOREIGN KEY (Member_ID) REFERENCES Member(Member_ID)
        ON DELETE CASCADE
);

CREATE TABLE External_Visitor (
    Member_ID  INT PRIMARY KEY,
    IC_Number  VARCHAR(30) NOT NULL UNIQUE,
    -- other attributes...
    FOREIGN KEY (Member_ID) REFERENCES Member(Member_ID)
        ON DELETE CASCADE
);
\end{sql}

Here, entity integrity is maintained because:

\begin{itemize}

\item
  Each subtype row has exactly one \texttt{Member\_ID} that is non-null
  and unique.
\item
  \texttt{Student\_ID}, \texttt{Staff\_ID}, and \texttt{IC\_Number} are
  declared \texttt{UNIQUE}, so they also behave as candidate keys within
  their respective tables, preventing two students from sharing the same
  student ID, etc.
\end{itemize}

\subsubsection{Weak Entities and Composite Primary
Keys}\label{weak-entities-and-composite-primary-keys}

Some tables represent relationship entities that depend on other
entities and use composite primary keys to ensure uniqueness:

\begin{sql}
CREATE TABLE Reservation_Equipments (
    Reservation_ID INT NOT NULL,
    Equipment_ID   INT NOT NULL,
    Quantity       INT NOT NULL,
    PRIMARY KEY (Reservation_ID, Equipment_ID)
);

CREATE TABLE Session_Enrollment (
    Reservation_ID INT NOT NULL,
    Member_ID      INT NOT NULL,
    PRIMARY KEY (Reservation_ID, Member_ID)
);

CREATE TABLE Member_Phone (
    Member_ID     INT NOT NULL,
    Phone_Number  VARCHAR(20) NOT NULL,
    PRIMARY KEY (Member_ID, Phone_Number)
);
\end{sql}

\begin{itemize}

\item
  In \texttt{Reservation\_Equipments}, the composite primary key
  prevents the same equipment from being listed twice for the same
  reservation.
\item
  In \texttt{Session\_Enrollment}, it prevents the same member from
  being enrolled multiple times in the same training session.
\item
  In \texttt{Member\_Phone}, it prevents duplicate phone numbers for the
  same member.
\end{itemize}

In all cases, the composite key columns are defined as
\texttt{NOT\ NULL}, maintaining entity integrity for these relationship
entities.

\subsection{Referential Integrity}\label{referential-integrity}

Referential integrity ensures that foreign key values always refer to existing
rows in the parent table and that deletions/updates do not create orphan
records. The formal reasoning behind referential integrity and its implications
across relational schemas has been extensively analyzed in recent work \citep{Kenig_2022}.

\subsubsection{Supertype--Subtype Foreign
Keys}\label{supertypesubtype-foreign-keys}

Each subtype references \texttt{Member} using a foreign key with
cascading delete:

\begin{sql}
ALTER TABLE Student
ADD CONSTRAINT fk_student_member
FOREIGN KEY (Member_ID)
REFERENCES Member(Member_ID)
ON DELETE CASCADE;

ALTER TABLE Staff
ADD CONSTRAINT fk_staff_member
FOREIGN KEY (Member_ID)
REFERENCES Member(Member_ID)
ON DELETE CASCADE;

ALTER TABLE External_Visitor
ADD CONSTRAINT fk_external_member
FOREIGN KEY (Member_ID)
REFERENCES Member(Member_ID)
ON DELETE CASCADE;
\end{sql}

\begin{itemize}

\item
  A row in \texttt{Student}, \texttt{Staff}, or
  \texttt{External\_Visitor} cannot exist without a corresponding
  \texttt{Member}.
\item
  When a \texttt{Member} is deleted, the related subtype row is
  automatically removed to avoid orphaned subtype records.
\end{itemize}

\subsubsection{Reservations, Bookings, and
Members/Facilities}\label{reservations-bookings-and-membersfacilities}

In the logical design, \texttt{Reservation} is a supertype for specific
reservation types such as \texttt{Booking} and
\texttt{Training\_Session}. Normal facility bookings are stored in
\texttt{Booking} and linked back to members and facilities as follows:

\begin{sql}
CREATE TABLE Reservation (
    Reservation_ID   INT PRIMARY KEY,
    Facility_ID      INT NOT NULL,
    Reservation_Date DATE NOT NULL,
    Start_Time       TIME NOT NULL,
    End_Time         TIME NOT NULL,
    FOREIGN KEY (Facility_ID) REFERENCES Facility(Facility_ID)
);

CREATE TABLE Booking (
    Reservation_ID  INT PRIMARY KEY,
    Member_ID       INT NOT NULL,
    Booking_Status  VARCHAR(20) NOT NULL,
    FOREIGN KEY (Reservation_ID) REFERENCES Reservation(Reservation_ID)
        ON DELETE CASCADE,
    FOREIGN KEY (Member_ID) REFERENCES Member(Member_ID)
);
\end{sql}

This ensures that:

\begin{itemize}

\item
  Every booking row corresponds to exactly one generic reservation, which
  already includes the facility and time slot information.
\item
  A booking cannot reference a non-existing member.
\item
  The facility association is managed at the \texttt{Reservation} level,
  ensuring that both \texttt{Booking} and \texttt{Training\_Session} 
  (which also inherits from \texttt{Reservation}) can utilize facilities
  consistently.
\item
  Deleting a reservation cascades to delete its associated booking,
  preventing orphan bookings.
\end{itemize}

\subsubsection{Training Sessions and
Coaches}\label{training-sessions-and-coaches}

Training sessions are another specialization of \texttt{Reservation} and
are taught by exactly one coach:

\begin{sql}
CREATE TABLE Training_Session (
    Reservation_ID INT PRIMARY KEY,
    Coach_ID       INT NOT NULL,
    Max_Capacity   INT NOT NULL,
    FOREIGN KEY (Reservation_ID) REFERENCES Reservation(Reservation_ID)
        ON DELETE CASCADE,
    FOREIGN KEY (Coach_ID) REFERENCES Coach(Coach_ID)
        ON DELETE RESTRICT
);
\end{sql}

Referential integrity here ensures that:

\begin{itemize}

\item
  A training session cannot exist unless its base \texttt{Reservation}
  exists.
\item
  Each session must be linked to a valid \texttt{Coach}.
\item
  The \texttt{ON\ DELETE\ RESTRICT} on \texttt{Coach} prevents deleting
  a coach while there are still sessions assigned to them, unless
  handled explicitly by the application or by reassigning the sessions.
\end{itemize}

\subsubsection{Equipment Assignments to
Reservations}\label{equipment-assignments-to-reservations}

The \texttt{Reservation\_Equipments} table connects reservations and
equipment:

\begin{sql}
ALTER TABLE Reservation_Equipments
ADD CONSTRAINT fk_res_eq_reservation
FOREIGN KEY (Reservation_ID)
REFERENCES Reservation(Reservation_ID)
ON DELETE CASCADE;

ALTER TABLE Reservation_Equipments
ADD CONSTRAINT fk_res_eq_equipment
FOREIGN KEY (Equipment_ID)
REFERENCES Equipment(Equipment_ID)
ON DELETE RESTRICT;
\end{sql}

\begin{itemize}

\item
  A row in \texttt{Reservation\_Equipments} cannot exist without both a
  valid \texttt{Reservation} and a valid \texttt{Equipment}.
\item
  If a reservation is deleted, its assigned equipment rows are also
  deleted.
\end{itemize}

\subsubsection{Session Enrollments}\label{session-enrollments}

\texttt{Session\_Enrollment} links members to training sessions:

\begin{sql}
ALTER TABLE Session_Enrollment
ADD CONSTRAINT fk_enrollment_reservation
FOREIGN KEY (Reservation_ID)
REFERENCES Training_Session(Reservation_ID)
ON DELETE CASCADE;

ALTER TABLE Session_Enrollment
ADD CONSTRAINT fk_enrollment_member
FOREIGN KEY (Member_ID)
REFERENCES Member(Member_ID)
ON DELETE RESTRICT;
\end{sql}

This guarantees that:

\begin{itemize}

\item
  Enrollments only exist for valid training sessions and members.
\item
  When a training session is deleted, related enrollments are removed
  automatically.
\end{itemize}

\subsubsection{Maintenance and Maintained
Resources}\label{maintenance-and-maintained-resources}

Each maintenance record applies either to a facility or to an equipment
item. We enforce referential integrity via optional foreign keys:

\begin{sql}
CREATE TABLE Maintenance (
    Maintenance_ID INT PRIMARY KEY,
    Equipment_ID   INT NULL,
    Facility_ID    INT NULL,
    Status         VARCHAR(20) NOT NULL,
    -- other attributes...
    -- Note: Each record represents maintenance of ONE equipment instance or facility.
    -- Multiple equipment items require multiple maintenance records.
    FOREIGN KEY (Equipment_ID) REFERENCES Equipment(Equipment_ID)
        ON DELETE RESTRICT,
    FOREIGN KEY (Facility_ID) REFERENCES Facility(Facility_ID)
        ON DELETE RESTRICT,
    CHECK (
        (Equipment_ID IS NOT NULL AND Facility_ID IS NULL) OR
        (Equipment_ID IS NULL AND Facility_ID IS NOT NULL)
    )
);
\end{sql}

\begin{itemize}

\item
  A maintenance record must point to exactly one existing equipment or
  facility.
\item
  The XOR \texttt{CHECK} constraint (discussed again in domain
  integrity) guarantees that both foreign keys cannot be set
  simultaneously or both be \texttt{NULL}.
\end{itemize}

\subsubsection{Visitor Applications}\label{visitor-applications}

\texttt{Visitor\_Application} records are created by prospective visitors
who are not yet members of the system:

\begin{sql}
CREATE TABLE Visitor_Application (
    Application_ID   INT PRIMARY KEY,
    First_Name       VARCHAR(50) NOT NULL,
    Last_Name        VARCHAR(50) NOT NULL,
    IC_Number        VARCHAR(20) NOT NULL UNIQUE,
    Application_Date DATE NOT NULL,
    Status           VARCHAR(20) NOT NULL,
    Approved_By      INT NULL,
    Approval_Date    DATE NULL,
    Reject_Reason    TEXT NULL,
    Created_Member_ID INT NULL,
    FOREIGN KEY (Approved_By) REFERENCES Staff(Member_ID)
        ON DELETE SET NULL,
    FOREIGN KEY (Created_Member_ID) REFERENCES Member(Member_ID)
        ON DELETE SET NULL
);
\end{sql}

\begin{itemize}
\item
  The applicant's basic information (\texttt{First\_Name},
  \texttt{Last\_Name}, \texttt{IC\_Number}) is stored directly in this
  table. At the time of application submission, the applicant is not yet
  a \texttt{Member}.
\item
  \texttt{IC\_Number} is marked as \texttt{UNIQUE} to prevent duplicate
  applications from the same person.
\item
  The \texttt{Approved\_By} field references \texttt{Staff} to track which
  staff member processed the application. If the staff member is deleted,
  \texttt{Approved\_By} is set to \texttt{NULL} to preserve the application
  history.
\item
  \texttt{Created\_Member\_ID} links to the \texttt{Member} record created
  after approval. For pending or rejected applications, this field is
  \texttt{NULL}. Upon approval, the staff creates a new \texttt{Member}
  and \texttt{External\_Visitor} record, then updates this field to
  establish the traceability link. If the member record is later deleted,
  this field is set to \texttt{NULL}, but the application record is preserved
  for audit purposes.
\end{itemize}

\subsubsection{Visitor Application Contact Details}

The multi-valued contact information is stored in separate weak entity tables:

\begin{sql}
CREATE TABLE Visitor_Application_Phone (
    Application_ID INT NOT NULL,
    Phone_Number   VARCHAR(20) NOT NULL,
    PRIMARY KEY (Application_ID, Phone_Number),
    FOREIGN KEY (Application_ID) REFERENCES Visitor_Application(Application_ID)
        ON DELETE CASCADE
);

CREATE TABLE Visitor_Application_Email (
    Application_ID  INT NOT NULL,
    Email_Address   VARCHAR(100) NOT NULL,
    PRIMARY KEY (Application_ID, Email_Address),
    FOREIGN KEY (Application_ID) REFERENCES Visitor_Application(Application_ID)
        ON DELETE CASCADE
);
\end{sql}

These tables allow each application to have multiple phone numbers and email
addresses. The composite primary keys prevent duplicate entries for the same
application. When an application is deleted, its contact details are
automatically removed via \texttt{ON DELETE CASCADE}.

\subsubsection{Member Phone Numbers}\label{member-phone-numbers}

Finally, \texttt{Member\_Phone} enforces referential integrity to
\texttt{Member}:

\begin{sql}
ALTER TABLE Member_Phone
ADD CONSTRAINT fk_phone_member
FOREIGN KEY (Member_ID)
REFERENCES Member(Member_ID)
ON DELETE CASCADE;
\end{sql}

This prevents phone numbers from being associated with non-existing
members and automatically cleans up phone records when a member is
deleted.

\subsection{Domain Integrity}\label{domain-integrity}

Domain integrity restricts the values that can be stored in each column
through data types, length limits, and \texttt{CHECK} constraints.

\subsubsection{Data Types and Length
Constraints}\label{data-types-and-length-constraints}

We choose SQL data types that reflect the meaning of each attribute:

\begin{itemize}

\item
  \texttt{DATE}, \texttt{TIME}, \texttt{DATETIME} for temporal
  attributes such as reservation start/end time and maintenance dates.
\item
  \texttt{INT} for identifiers, quantities, and capacities.
\item
  \texttt{VARCHAR(n)} for names, phone numbers, and descriptive fields,
  with reasonable length limits (e.g., \texttt{VARCHAR(50)} for names,
  \texttt{VARCHAR(20)} for phone numbers) to prevent excessively long
  inputs.
\end{itemize}

Example:

\begin{sql}
CREATE TABLE Facility (
    Facility_ID   INT PRIMARY KEY,
    Facility_Name VARCHAR(100) NOT NULL,
    Capacity      INT NOT NULL,
    Status        VARCHAR(20) NOT NULL
);
\end{sql}

\subsubsection{Enumerated Status Values}\label{enumerated-status-values}

Several attributes are restricted to predefined sets of values using
\texttt{CHECK} constraints:

\begin{sql}
ALTER TABLE Member
ADD CONSTRAINT chk_member_status
CHECK (Membership_Status IN ('Active', 'Inactive', 'Pending_Approval'));

ALTER TABLE Booking
ADD CONSTRAINT chk_booking_status
CHECK (Booking_Status IN ('Pending', 'Confirmed', 'Cancelled'));

ALTER TABLE Facility
ADD CONSTRAINT chk_facility_status
CHECK (Status IN ('Available', 'Under_Maintenance', 'Unavailable'));

ALTER TABLE Equipment
ADD CONSTRAINT chk_equipment_status
CHECK (Status IN ('Available', 'In_Use', 'Under_Maintenance', 'Damaged'));

ALTER TABLE Maintenance
ADD CONSTRAINT chk_maintenance_status
CHECK (Status IN ('Scheduled', 'In_Progress', 'Completed', 'Cancelled'));

ALTER TABLE Visitor_Application
ADD CONSTRAINT chk_application_status
CHECK (Status IN ('Pending', 'Approved', 'Rejected'));
\end{sql}

These constraints ensure that status columns only store meaningful,
valid values consistent with the business rules.

\subsubsection{Numeric ranges and inventory
consistency}\label{numeric-ranges-and-inventory-consistency}

Numeric attributes are constrained to realistic ranges:

\begin{sql}
ALTER TABLE Facility
ADD CONSTRAINT chk_facility_capacity
CHECK (Capacity > 0);

ALTER TABLE Equipment
ADD CONSTRAINT chk_equipment_quantity
CHECK (Total_Quantity >= 0);

-- Note: Available_Quantity is a derived attribute (not stored in the table),
-- so no constraint is defined for it. It is calculated dynamically as:
-- Total_Quantity - (quantity in active reservations + count of active maintenance records)

ALTER TABLE Reservation_Equipments
ADD CONSTRAINT chk_res_eq_quantity
CHECK (Quantity > 0);
\end{sql}

This prevents negative capacities or quantities, and ensures that
available equipment does not exceed the total quantity.

\subsubsection{Temporal Consistency}\label{temporal-consistency}

We enforce logical ordering of time-related attributes. For example, a
reservation must end after it starts:

\begin{sql}
ALTER TABLE Reservation
ADD CONSTRAINT chk_reservation_time
CHECK (Start_Time < End_Time);
\end{sql}

Optionally, we may also restrict new reservations so that they are not
created in the past:

\begin{sql}
-- Optional constraint, depending on operational policy
-- Note: Since we use DATE + TIME separation, we check the date component
ALTER TABLE Reservation
ADD CONSTRAINT chk_reservation_date_not_past
CHECK (Reservation_Date >= CURDATE());

-- For same-day reservations, time validation would need to be done
-- at the application layer, as:
-- CHECK ((Reservation_Date > CURDATE()) OR 
--        (Reservation_Date = CURDATE() AND Start_Time >= CURTIME()))
-- This complex check is better handled in application logic
\end{sql}

\subsubsection{Business Rule Constraints}\label{business-rule-constraints}

Based on the assumptions defined in \cref{assumptions}, we enforce the following
business rules at the database level where feasible:

\paragraph{Maximum Booking Duration (3 hours)}

Each member's single booking session cannot exceed three hours. This rule
can be enforced directly using a \texttt{CHECK} constraint on the
\texttt{Reservation} table:

\begin{sql}
ALTER TABLE Reservation
ADD CONSTRAINT chk_reservation_max_duration
CHECK (TIMESTAMPDIFF(HOUR, 
    CONCAT(Reservation_Date, ' ', Start_Time),
    CONCAT(Reservation_Date, ' ', End_Time)
) <= 3);
\end{sql}

This constraint calculates the time difference between start and end times
and ensures it does not exceed 3 hours.

\paragraph{Maximum Advance Booking (1 week)}

Members may book facilities up to one week in advance. This constraint
prevents reservations from being made too far in the future:

\begin{sql}
ALTER TABLE Reservation
ADD CONSTRAINT chk_reservation_max_advance
CHECK (Reservation_Date <= DATE_ADD(CURDATE(), INTERVAL 7 DAY));
\end{sql}

\paragraph{Maximum Pending Bookings per Member}

The rule "each member may have at most 2 pending bookings at any given time"
requires counting rows across the \texttt{Booking} table filtered by
\texttt{Member\_ID} and \texttt{Booking\_Status = 'Pending'}. Since MySQL
CHECK constraints do not support subqueries, this rule \textbf{must be
enforced at the application layer} or via a \texttt{BEFORE INSERT} trigger.

Application-level enforcement example:

\begin{sql}
-- Before inserting a new Booking, the application should execute:
SELECT COUNT(*)
FROM Booking
WHERE Member_ID = ? AND Booking_Status = 'Pending';

-- If count >= 2, reject the new booking request
\end{sql}

Alternatively, a database trigger can enforce this rule:

\begin{sql}
DELIMITER $$
CREATE TRIGGER trg_check_pending_bookings
BEFORE INSERT ON Booking
FOR EACH ROW
BEGIN
    DECLARE pending_count INT;
    
    SELECT COUNT(*) INTO pending_count
    FROM Booking
    WHERE Member_ID = NEW.Member_ID 
      AND Booking_Status = 'Pending';
    
    IF pending_count >= 2 THEN
        SIGNAL SQLSTATE '45000'
        SET MESSAGE_TEXT = 'Member already has 2 pending bookings';
    END IF;
END$$
DELIMITER ;
\end{sql}

This separation of concerns—simple constraints at the database level,
complex counting logic at the application or trigger level—ensures that
the system remains maintainable while enforcing all necessary business
rules.

In practice, this constraint would typically be enforced at the
application layer or only for newly created records, to allow importing
historical data if needed.

\subsubsection{XOR Constraint for
Maintenance}\label{xor-constraint-for-maintenance}

As shown in Section 2.6, we use a \texttt{CHECK} constraint in
\texttt{Maintenance} to ensure that each maintenance record refers to
exactly one type of resource (either an equipment item or a facility):

\begin{sql}
ALTER TABLE Maintenance
ADD CONSTRAINT chk_maintenance_target
CHECK (
    (Equipment_ID IS NOT NULL AND Facility_ID IS NULL) OR
    (Equipment_ID IS NULL AND Facility_ID IS NOT NULL)
);
\end{sql}

This enforces the business rule that a maintenance task cannot
simultaneously target both a facility and an equipment item, and cannot
be left without a target.

\subsubsection{Multi-Valued Attributes}\label{multi-valued-attributes}

The multi-valued attribute \texttt{Phone\_Number} of \texttt{Member} is
transformed into a separate table \texttt{Member\_Phone}. Domain
integrity is enforced by:

\begin{itemize}

\item
  Choosing \texttt{VARCHAR(20)} for \texttt{Phone\_Number} to allow
  different phone formats.
\item
  Using the composite primary key \texttt{(Member\_ID,\ Phone\_Number)}
  to avoid duplicates.
\item
  Applying a simple pattern check if needed (implementation-dependent),
  e.g.:
\end{itemize}

\begin{sql}
-- Example pattern check (syntax depends on the SQL dialect)
ALTER TABLE Member_Phone
ADD CONSTRAINT chk_phone_format
CHECK (Phone_Number <> '');  -- or a more specific pattern using LIKE/REGEXP
\end{sql}

\subsection{Summary}\label{summary}

Our design enforces:

\begin{itemize}

\item
  \textbf{Entity integrity} by defining appropriate primary keys for all
  strong and weak entities, using non-null keys and \texttt{UNIQUE}
  constraints for natural candidate keys such as \texttt{Student\_ID},
  \texttt{Staff\_ID}, and \texttt{IC\_Number}.
\item
  \textbf{Referential integrity} by specifying foreign keys that
  accurately reflect the relationships in the EERD (including
  supertype--subtype links, bookings and training sessions derived from
  reservations, maintenance targets, and enrollment/assignment tables),
  with suitable actions on delete.
\item
  \textbf{Domain integrity} by carefully selecting data types and
  lengths and by using \texttt{CHECK} constraints on enumerated
  statuses, numeric ranges, time ordering, and XOR conditions.
\end{itemize}

Together, these constraints ensure that the database state remains
consistent with the business rules of the sports facility management
system.

\section{Database Transaction}\label{task-7-acid}

For Task 7, we analyze how our \textbf{actual system design} (EERD,
relational schema, constraints, and access control) works together with
the DBMS transaction mechanism to maintain all four ACID properties for
key operations such as creating bookings, borrowing equipment, enrolling
in training sessions, scheduling maintenance, and approving
registrations.

\subsection{Atomicity}\label{atomicity}

Atomicity means that a business operation that consists of several
database statements is treated as one indivisible transaction:
it either completes fully or is entirely rolled back.

In our system, we treat each high‑level business action as a single
transaction. For example, when a member books a court and
borrows equipment at the same time, the following steps logically belong
to one transaction:

\begin{itemize}

\item
  Insert a new row into \texttt{Reservation} (supertype) and its subtype
  \texttt{Booking}.
\item
  Insert one or more rows into \texttt{Reservation\_Equipments} to
  record which equipment is borrowed and in what quantity.
\item
  Rely on the derived \texttt{Available\_Quantity} (calculated from
  \texttt{Equipment.Total\_Quantity}, \texttt{Reservation\_Equipments},
  and active \texttt{Maintenance} records) when checking availability,
  rather than updating a stored column.
\end{itemize}

In SQL‑style pseudocode, this is wrapped in a transaction block:

\begin{sql}
START TRANSACTION;
  -- insert Reservation and Booking
  -- insert Reservation_Equipments rows
  -- (Available_Quantity is a derived attribute, no update needed)
COMMIT; 
\end{sql}

If any step fails (for example, a foreign key violation, a
\texttt{CHECK} constraint failure, or an application‑level validation
error), the application will execute \texttt{ROLLBACK}, causing
all inserts and updates in this block to be undone. Thus, we
never end up in a state where the booking exists but the equipment
allocation was not recorded, or vice versa. The same pattern applies to
other multi‑step actions such as cancelling a booking (update booking
status and delete \texttt{Reservation\_Equipments} records, which
automatically frees equipment) or approving a
\texttt{Visitor\_Application} and creating the corresponding
\texttt{Member} and subtype row (\texttt{Student}, \texttt{Staff}, or
\texttt{External\_Visitor}).

\subsection{Consistency}\label{consistency}

Consistency means that every committed transaction takes the database
from one valid state to another, where all constraints and
business rules are satisfied. In our design, consistency is mainly
enforced through the integrity constraints defined in Task 6, combined
with transactional execution.

Key mechanisms include:

\begin{itemize}

\item
  \textbf{Entity integrity:} Primary keys on tables such as
  \texttt{Member}, \texttt{Coach}, \texttt{Facility},
  \texttt{Equipment}, \texttt{Reservation}, and \texttt{Maintenance}
  ensure that no duplicate or \texttt{NULL} identifiers are inserted.
\item
  \textbf{Referential integrity:} Foreign keys (for example, from
  \texttt{Booking} to \texttt{Reservation}, from
  \texttt{Reservation\_Equipments} to both \texttt{Reservation} and
  \texttt{Equipment}, from subtypes to \texttt{Member}) ensure that a
  transaction cannot create orphan records. If a booking references a
  non‑existent member or facility, the insert fails and the transaction
  is rolled back.
\item
  \textbf{Domain integrity:} \texttt{CHECK} constraints such as
  \texttt{Start\_Time\ \textless{}\ End\_Time},
  \texttt{Capacity\ \textgreater{}\ 0}, and the status enumerations
  (e.g.~\texttt{Booking\_Status\ IN\ (\textquotesingle{}Pending\textquotesingle{},\ \textquotesingle{}Confirmed\textquotesingle{},\ \textquotesingle{}Cancelled\textquotesingle{})})
  prevent invalid attribute values from being committed.
\item
  \textbf{Business rules on conflicts:} For facilities, we use indexed
  time fields
  (\texttt{Reservation(\\ Facility\_ID,\ Date,\ Start\_Time,\ End\_Time)}
  and \texttt{Maintenance(Facility\_ID,\ Scheduled\_Date)}) together
  with conflict‑checking logic (e.g.~a stored procedure that checks for
  overlapping time ranges, or a view such as
  \texttt{v\_facility\_occupancy}). If a new reservation would overlap
  with an existing booking or maintenance slot, the procedure signals an
  error and the transaction is not committed.
\end{itemize}

Because all these constraints are checked within the
transaction, any attempt to violate a rule causes that transaction to
fail and roll back. As a result, the database is always left in a state
that is consistent with the conceptual EERD and the real‑world rules of
the sports complex.

\subsection{Isolation}\label{isolation}

Isolation ensures that concurrent transactions do not interfere with
each other in a way that produces incorrect results. Each transaction
should behave as if it is running alone, even when many users are
booking facilities or borrowing equipment at the same time.

Consider the case where there are 5 rackets available. User A tries to
borrow 4, and at nearly the same time User B tries to borrow 3. If these
operations are not properly isolated, both might read ``5 available''
and we could end up with \texttt{Available\_Quantity\ =\ -2}. To avoid
this, we rely on the DBMS's isolation mechanism (for example, the
default \texttt{READ\ COMMITTED} isolation level with row‑level locking)
and explicitly treat stock updates as critical sections:

\begin{sql}
START TRANSACTION;
    -- Lock the equipment row and calculate Available_Quantity
    SELECT 
        Total_Quantity - IFNULL(
            (SELECT SUM(Quantity) FROM Reservation_Equipments 
             WHERE Equipment_ID = ? AND ...), 0
        ) - IFNULL(
            (SELECT COUNT(*) FROM Maintenance 
             WHERE Equipment_ID = ? 
             AND Status IN ('Scheduled', 'In_Progress')), 0
        ) AS Available_Quantity
    FROM Equipment
    WHERE Equipment_ID = ?
    FOR UPDATE;

    -- check if Available_Quantity is sufficient
    -- if sufficient, insert Reservation_Equipments
COMMIT;
\end{sql}

The \texttt{FOR\ UPDATE} clause (or the equivalent locking mechanism in
the chosen DBMS) locks the selected equipment row so that only one
transaction can modify the equipment allocation at a time. The second
transaction will have to wait until the first commits, and will then
calculate the updated \texttt{Available\_Quantity} based on the new
state. Combined with application‑level validation that prevents negative
or over‑allocated quantities, this prevents lost updates and over‑booking
of equipment.

For facility bookings and training sessions, similar isolation is
achieved: conflict‑checking queries run inside the same transaction that
inserts the new reservation. If isolation is not strong enough in the
default setting, we can raise the isolation level (e.g.~to
\texttt{SERIALIZABLE}) or use stricter locking on the
\texttt{Reservation} rows for a given facility and time range to avoid
double bookings and phantom conflicts.

\subsection{Durability}\label{durability}

Durability means that once a transaction has been committed,
its effects are permanent, even if there is a system crash or power
failure immediately afterwards.

In our system, durability is provided by the underlying DBMS. When the
application issues a \texttt{COMMIT} for a transaction (for example,
after successfully creating a booking, enrolling a member into a
training session, or marking a maintenance job as \texttt{Completed}),
the DBMS:

\begin{itemize}

\item
  First writes the changes to the transaction log (often using
  a write‑ahead logging mechanism).
\item
  Then applies the changes to the data files on disk.
\end{itemize}

If a crash occurs, the DBMS replays the log during recovery to
redo committed transactions and undo any incomplete
ones, so that confirmed bookings and maintenance records are not lost.
This technical durability is complemented by operational practices from
our access‑control design: the System Administrator (DBA role) is
responsible for regular backups and recovery procedures, ensuring that
even in the case of hardware failure, committed data can be restored
from backups and logs.

Therefore, once a member receives confirmation that their booking has
been successfully stored (i.e.~the transaction has committed), they can
be confident that the booking will remain in the system despite
unexpected failures.

\newpage
\bibliography{ref}



% 添加 Marking Rubric 到目录(无编号,无点线,无页码)
\label{LastMainPage}  % 标记正文最后一页
\clearpage
\pagenumbering{gobble}  % 停止页码显示

\phantomsection
\addtocontents{toc}{\cftpagenumbersoff{section}}
\addcontentsline{toc}{section}{Turnitin Percentage Page}
\includepdf[pages=2, fitpaper=true]{report.pdf}

\phantomsection
\addtocontents{toc}{\cftpagenumbersoff{section}}
\addcontentsline{toc}{section}{Marking Rubric}
\includepdf[pages=-, fitpaper=true]{rubric.pdf}



\end{document}
