\section{SQL}

This section demonstrates the practical implementation of the University Sports
Complex Management System database. It covers the creation of the remaining
tables, data population, and various data retrieval operations using SQL.

\subsection{Table Creation}

Following the initial examples in Task 1, we proceeded to create the remaining
tables to complete the database schema. These tables include entities for
different member types (Student, Staff, External Visitor), contact information,
coaching staff, facility details, and various transaction tables like
maintenance records and session enrollments.

The following screenshots demonstrate the successful execution of the
\texttt{CREATE TABLE} commands for these entities.

\begin{figure}[H]
    \centering
    \includegraphics[width=\textwidth]{figure/task2_sc/screenshot_create_tables_part1.png}
    \caption{CREATE TABLE execution (\texttt{student}, \texttt{staff}, \texttt{external\_visitor})}
    \label{fig:create_tables_1}
\end{figure}

\begin{figure}[H]
    \centering
    \includegraphics[width=\textwidth]{figure/task2_sc/screenshot_create_tables_part2.png}
    \caption{CREATE TABLE execution (\texttt{member\_email}, \texttt{member\_phone}, \texttt{coach\_email}, \texttt{coach\_phone})}
    \label{fig:create_tables_2}
\end{figure}

\begin{figure}[H]
    \centering
    \includegraphics[width=\textwidth]{figure/task2_sc/screenshot_create_tables_part3.png}
    \caption{CREATE TABLE execution (\texttt{facility}, \texttt{coach}, \texttt{maintenance})}
    \label{fig:create_tables_3}
\end{figure}

\begin{figure}[H]
    \centering
    \includegraphics[width=\textwidth]{figure/task2_sc/screenshot_create_tables_part4.png}
    \caption{CREATE TABLE execution (\texttt{visitor\_application}, \texttt{training\_session}, \texttt{session\_enrollment}, \texttt{reservation\_equipments})}
    \label{fig:create_tables_4}
\end{figure}

\subsection{Data Population}

To ensure the database is functional for testing and queries, we populated each
table with at least 6 rows of data (meeting the required 6--10 rows where
practical). In a few cases, additional rows were inserted to support more
comprehensive trigger testing scenarios in Task 3. The screenshots below provide
evidence of successful data population and a row-count summary for key tables.

\begin{figure}[H]
    \centering
    \includegraphics[width=\textwidth]{figure/task2_sc/screenshot_insert.png}
    \caption{Data population execution evidence: representative \texttt{INSERT}/\texttt{UPDATE} statements executed successfully across the schema.}
    \label{fig:data_population_execution}
\end{figure}

\begin{figure}[H]
    \centering
    \includegraphics[width=0.8\textwidth]{figure/task2_sc/screenshot_data_population_count.png}
    \caption{Verification of Data Population (Row Counts)}
    \label{fig:data_count}
\end{figure}

\subsection{Data Retrieval and Manipulation}

We executed various SQL commands across six categories to demonstrate data
retrieval and manipulation capabilities.

\subsubsection{Filtering}

This category demonstrates the use of \texttt{LIKE}, \texttt{BETWEEN}, and
\texttt{IN} clauses to filter results.

\textbf{Query 1: Filtering by Pattern (LIKE)} \\
Find all facilities related to 'Tennis' that are currently available.
\begin{sql}[caption={Filtering with LIKE}]
SELECT Facility_Name, Type, Status
FROM facility
WHERE Type LIKE '%Tennis%' AND Status='Available';
\end{sql}

\begin{figure}[H]
    \centering
    \includegraphics[width=0.4\textwidth]{figure/task2_sc/query1.png}
    \caption{Output of Query 1}
    \label{fig:query1}
\end{figure}

\textbf{Query 2: Filtering by Range (BETWEEN)} \\
Retrieve reservations made for the month of December 2025.
\begin{sql}[caption={Filtering with BETWEEN}]
SELECT Reservation_ID, Reservation_Date
FROM reservation
WHERE Reservation_Date BETWEEN '2025-12-01' AND '2025-12-31';
\end{sql}

\begin{figure}[H]
    \centering
    \includegraphics[width=0.4\textwidth]{figure/task2_sc/query2.png}
    \caption{Output of Query 2}
    \label{fig:query2}
\end{figure}

\textbf{Query 3: Filtering by Set (IN)} \\
List bookings that are either 'Pending' or 'Confirmed'.
\begin{sql}[caption={Filtering with IN}]
SELECT Reservation_ID, Booking_Status
FROM booking
WHERE Booking_Status IN ('Pending','Confirmed');
\end{sql}

\begin{figure}[H]
    \centering
    \includegraphics[width=0.4\textwidth]{figure/task2_sc/query3.png}
    \caption{Output of Query 3}
    \label{fig:query3}
\end{figure}



\subsubsection{Aggregate Functions}

This category uses functions like \texttt{COUNT}, \texttt{SUM}, and \texttt{AVG} to summarize data.

\textbf{Query 4: Counting Records (COUNT)} \\
Count the number of facilities for each type.
\begin{sql}[caption={Aggregation with COUNT}]
SELECT Type, COUNT(*) AS facility_count
FROM facility
GROUP BY Type;
\end{sql}

\begin{figure}[H]
    \centering
    \includegraphics[width=0.4\textwidth]{figure/task2_sc/query4.png}
    \caption{Output of Query 4}
    \label{fig:query4}
\end{figure}

\textbf{Query 5: Summing Values (SUM)} \\
Calculate the total quantity of equipment reserved for each reservation ID.
\begin{sql}[caption={Aggregation with SUM}]
SELECT Reservation_ID, SUM(Quantity) AS total_equipment
FROM reservation_equipments
GROUP BY Reservation_ID;
\end{sql}

\begin{figure}[H]
    \centering
    \includegraphics[width=0.4\textwidth]{figure/task2_sc/query5.png}
    \caption{Output of Query 5}
    \label{fig:query5}
\end{figure}

\textbf{Query 6: Averaging Values (AVG)} \\
Calculate the average maximum capacity of training sessions for each coach.
\begin{sql}[caption={Aggregation with AVG}]
SELECT Coach_ID, AVG(Max_Capacity) AS avg_capacity
FROM training_session
GROUP BY Coach_ID;
\end{sql}

\begin{figure}[H]
    \centering
    \includegraphics[width=0.4\textwidth]{figure/task2_sc/query6.png}
    \caption{Output of Query 6}
    \label{fig:query6}
\end{figure}



\subsubsection{Limit / Sorting}

This category demonstrates sorting results with \texttt{ORDER BY} and restricting output with \texttt{LIMIT}.

\textbf{Query 7: Ordering and Limiting} \\
List the top 3 available facilities with the highest capacity.
\begin{sql}[caption={Sorting and Limiting results}]
SELECT Facility_Name, Capacity 
FROM facility
WHERE Status='Available' 
ORDER BY Capacity DESC 
LIMIT 3;
\end{sql}

\begin{figure}[H]
    \centering
    \includegraphics[width=0.4\textwidth]{figure/task2_sc/query7.png}
    \caption{Output of Query 7}
    \label{fig:query7}
\end{figure}

\textbf{Query 8: Multi-level Sorting} \\
Show recent reservations ordered by date and time in descending order.
\begin{sql}[caption={Multi-column Sorting}]
SELECT Reservation_ID, Reservation_Date, Start_Time 
FROM reservation
ORDER BY Reservation_Date DESC, Start_Time DESC 
LIMIT 5;
\end{sql}

\begin{figure}[H]
    \centering
    \includegraphics[width=0.4\textwidth]{figure/task2_sc/query8.png}
    \caption{Output of Query 8}
    \label{fig:query8}
\end{figure}

\textbf{Query 9: Sorting Applications} \\
List the 5 most recent visitor applications.
\begin{sql}[caption={Sorting Visitor Applications}]
SELECT Application_ID, First_Name, Status 
FROM visitor_application
ORDER BY Application_Date DESC 
LIMIT 5;
\end{sql}

\begin{figure}[H]
    \centering
    \includegraphics[width=0.4\textwidth]{figure/task2_sc/query9.png}
    \caption{Output of Query 9}
    \label{fig:query9}
\end{figure}



\clearpage

\subsubsection{Join Operators}

This category demonstrates \texttt{INNER JOIN}, \texttt{LEFT JOIN}, and \texttt{RIGHT JOIN} to combine data from multiple tables.

\textbf{Query 10: Inner Join} \\
Retrieve booking details including member name, facility name, and booking status.
\begin{sql}[caption={Inner Join Example}]
SELECT b.Reservation_ID, m.First_Name, f.Facility_Name, b.Booking_Status 
FROM booking b
JOIN member m ON b.Member_ID=m.Member_ID 
JOIN reservation r ON b.Reservation_ID=r.Reservation_ID 
JOIN facility f ON r.Facility_ID=f.Facility_ID;
\end{sql}

\begin{figure}[H]
    \centering
    \includegraphics[width=0.4\textwidth]{figure/task2_sc/query10.png}
    \caption{Output of Query 10}
    \label{fig:query10}
\end{figure}

\textbf{Query 11: Left Join} \\
List all training sessions and their coaches (if assigned), facilities, and capacity.
\begin{sql}[caption={Left Join Example}]
SELECT ts.Reservation_ID, c.First_Name, f.Facility_Name, 
       ts.Max_Capacity 
FROM training_session ts
LEFT JOIN coach c ON ts.Coach_ID=c.Coach_ID 
LEFT JOIN reservation r ON ts.Reservation_ID=r.Reservation_ID 
LEFT JOIN facility f ON r.Facility_ID=f.Facility_ID;
\end{sql}

\begin{figure}[H]
    \centering
    \includegraphics[width=0.4\textwidth]{figure/task2_sc/query11.png}
    \caption{Output of Query 11}
    \label{fig:query11}
\end{figure}

\textbf{Query 12: Right Join} \\
List equipment reservation-assignment rows, demonstrating \texttt{RIGHT JOIN} syntax.
\begin{sql}[caption={Right Join Example}]
SELECT e.Equipment_Name, re.Reservation_ID, re.Quantity 
FROM equipment e
RIGHT JOIN reservation_equipments re ON e.Equipment_ID=re.Equipment_ID;
\end{sql}

\begin{figure}[H]
    \centering
    \includegraphics[width=0.4\textwidth]{figure/task2_sc/query12.png}
    \caption{Output of Query 12}
    \label{fig:query12}
\end{figure}



\subsubsection{String / Arithmetic Operations}

This category performs calculations and string manipulations on retrieved data.

\textbf{Query 13: Arithmetic Expression} \\
Calculate the duration of reservations in hours.
\begin{sql}[caption={Arithmetic Operation}]
SELECT Reservation_ID, 
       TIMESTAMPDIFF(MINUTE, Start_Time, End_Time)/60 AS duration_hours 
FROM reservation;
\end{sql}

\begin{figure}[H]
    \centering
    \includegraphics[width=0.4\textwidth]{figure/task2_sc/query13.png}
    \caption{Output of Query 13}
    \label{fig:query13}
\end{figure}

\textbf{Query 14: String Concatenation} \\
Combine first and last names into a single full name column for members.
\begin{sql}[caption={String Concatenation}]
SELECT CONCAT(First_Name, ' ', Last_Name) AS member_name, 
       Membership_Status 
FROM member;
\end{sql}

\begin{figure}[H]
    \centering
    \includegraphics[width=0.4\textwidth]{figure/task2_sc/query14.png}
    \caption{Output of Query 14}
    \label{fig:query14}
\end{figure}

\textbf{Query 15: Complex Calculation} \\
Calculate the theoretical remaining quantity of equipment after reservations.
\begin{sql}[caption={Complex Arithmetic}]
SELECT re.Reservation_ID, re.Quantity, e.Total_Quantity, 
       (e.Total_Quantity - re.Quantity) AS remaining_theoretical 
FROM reservation_equipments re
JOIN equipment e ON re.Equipment_ID=e.Equipment_ID;
\end{sql}

\begin{figure}[H]
    \centering
    \includegraphics[width=0.4\textwidth]{figure/task2_sc/query15.png}
    \caption{Output of Query 15}
    \label{fig:query15}
\end{figure}



\subsubsection{Formatting}

This category focuses on formatting query output using aliases and concatenation for better readability.

\textbf{Query 16: Column Aliasing} \\
Rename columns for a clear facility location report.
\begin{sql}[caption={Column Aliasing}]
SELECT Facility_ID AS ID, Facility_Name AS Name, 
       CONCAT(Building,'-',Floor,'-',Room_Number) AS Location 
FROM facility;
\end{sql}

\begin{figure}[H]
    \centering
    \includegraphics[width=0.4\textwidth]{figure/task2_sc/query16.png}
    \caption{Output of Query 16}
    \label{fig:query16}
\end{figure}

\textbf{Query 17: Date Formatting} \\
Format reservation dates and create user-friendly booking summaries.
\begin{sql}[caption={Date Formatting and Concatenation}]
SELECT b.Reservation_ID AS BookingID, 
       CONCAT(m.First_Name,' ',m.Last_Name) AS Member, 
       DATE_FORMAT(r.Reservation_Date,'%Y-%m-%d') AS Date 
FROM booking b
JOIN member m ON b.Member_ID=m.Member_ID 
JOIN reservation r ON b.Reservation_ID=r.Reservation_ID;
\end{sql}

\begin{figure}[H]
    \centering
    \includegraphics[width=0.4\textwidth]{figure/task2_sc/query17.png}
    \caption{Output of Query 17}
    \label{fig:query17}
\end{figure}

\textbf{Query 18: Session Formatting} \\
Create a formatted schedule for training sessions.
\begin{sql}[caption={Result Set Formatting}]
SELECT ts.Reservation_ID AS SessionID, 
       CONCAT(c.First_Name,' ',c.Last_Name) AS CoachName, 
       CONCAT(r.Start_Time,'-',r.End_Time) AS TimeSlot 
FROM training_session ts
JOIN coach c ON ts.Coach_ID=c.Coach_ID 
JOIN reservation r ON ts.Reservation_ID=r.Reservation_ID;
\end{sql}

\begin{figure}[H]
    \centering
    \includegraphics[width=0.4\textwidth]{figure/task2_sc/query18.png}
    \caption{Output of Query 18}
    \label{fig:query18}
\end{figure}
